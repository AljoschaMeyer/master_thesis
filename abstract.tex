% !TEX root = main.tex

\thispagestyle{plain}
\begin{center}
    \Large
    \textbf{Range-Based Set Synchronization}
       
    \vspace{0.9cm}
    \textbf{Abstract}
\end{center}

Set reconciliation is the problem of computing the union between two sets stored on different machines. Set mirroring is the problem of updating one of these sets to be equal to the other. Optimal communication complexity requires transmitting a number of bits proportional to $n_{\triangle}$, the number of items occurring in exactly one of the two sets. While there are sophisticated set reconcilliation protocols achieving this optimum, they require computation time proportional to the size of the sets and computation space proportional to $n_{\triangle}$. Furthermore, they cannot be modified to solve the set mirroring problem.

Range-based synchronization is a simpler approach based on comparing hashes of ranges over the set and recursing if the two nodes arrive at different hashes, solving both the set reconciliation and the set mirroring problem. The number of bits transmitted is worse than the optimum by a factor of the logarithm of the size of the sets, but in exchange the computation time is only proportional to $n_{\triangle}$ and the computation space is constant.