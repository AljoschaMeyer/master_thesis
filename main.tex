\documentclass[12pt,twoside]{report}
\usepackage[utf8]{inputenc}

\usepackage[a4paper,width=150mm,top=25mm,bottom=25mm]{geometry}


\usepackage{amsmath}
\usepackage{amssymb}
\usepackage{cite}
%\usepackage{amsthm}

\usepackage{hyperref}
\usepackage{cleveref}

\usepackage{mathtools}
\DeclarePairedDelimiter\ceil{\lceil}{\rceil}
\DeclarePairedDelimiter\floor{\lfloor}{\rfloor}
\usepackage{enumitem}
\setlist[description]{leftmargin=\parindent,labelindent=\parindent}
\renewcommand{\epsilon}{\varepsilon}

\usepackage{macros}

\usepackage{tikz}
\usetikzlibrary{arrows.meta}
\usetikzlibrary{shapes}
\usetikzlibrary{calc}
\usetikzlibrary{math}
\usepackage{pgffor}

\usepackage{listings}
\usepackage{minted}

\title{
{Efficient Synchronization of Recursively Partitionable Data Structures}\\
{\large Technische Universität Berlin}
}
\author{Aljoscha Meyer}




\newcommand{\examplefp}[1]{#1}
\newcommand{\fpa}[0]{\examplefp{144}}
\newcommand{\fpb}[0]{\examplefp{194}}
\newcommand{\fpc}[0]{\examplefp{240}}
\newcommand{\fpd}[0]{\examplefp{245}}
\newcommand{\fpe}[0]{\examplefp{76}}
\newcommand{\fpf}[0]{\examplefp{221}}
\newcommand{\fpg}[0]{\examplefp{224}}
\newcommand{\fph}[0]{\examplefp{65}}
\newcommand{\fpcd}[0]{\examplefp{229}}
\newcommand{\fpacd}[0]{\examplefp{117}}
\newcommand{\fpefgh}[0]{\examplefp{74}}
\newcommand{\fpbcdh}[0]{\examplefp{232}}
\newcommand{\fpzero}[0]{\examplefp{0}}

%\newcommand{\examplea}[0]{a}
%\newcommand{\exampleb}[0]{b}
%\newcommand{\examplec}[0]{c}
%\newcommand{\exampled}[0]{d}
%\newcommand{\examplee}[0]{e}
%\newcommand{\examplef}[0]{f}
%\newcommand{\exampleg}[0]{g}
%\newcommand{\exampleh}[0]{h}
\newcommand{\examplea}[0]{\ensuremath{\mathrm{ape}}}
\newcommand{\exampleb}[0]{\ensuremath{\mathrm{bee}}}
\newcommand{\examplec}[0]{\ensuremath{\mathrm{cat}}}
\newcommand{\exampled}[0]{\ensuremath{\mathrm{doe}}}
\newcommand{\examplee}[0]{\ensuremath{\mathrm{eel}}}
\newcommand{\examplef}[0]{\ensuremath{\mathrm{fox}}}
\newcommand{\exampleg}[0]{\ensuremath{\mathrm{gnu}}}
\newcommand{\exampleh}[0]{\ensuremath{\mathrm{hog}}}
\newcommand{\examplei}[0]{\ensuremath{\examplea}}
\newcommand{\exampley}[0]{\ensuremath{\mathrm{yak}}}

\newcommand{\hexamplea}[0]{\ensuremath{\h(\mathrm{ape}})}
\newcommand{\hexampleb}[0]{\ensuremath{\h(\mathrm{bee}})}
\newcommand{\hexamplec}[0]{\ensuremath{\h(\mathrm{cat}})}
\newcommand{\hexampled}[0]{\ensuremath{\h(\mathrm{doe}})}
\newcommand{\hexamplee}[0]{\ensuremath{\h(\mathrm{eel}})}
\newcommand{\hexamplef}[0]{\ensuremath{\h(\mathrm{fox}})}
\newcommand{\hexampleg}[0]{\ensuremath{\h(\mathrm{gnu}})}
\newcommand{\hexampleh}[0]{\ensuremath{\h(\mathrm{hog}})}

\newcommand{\aux}[2]{
\begin{tabular}{ c }
{#1} \\ 
\hline
${#2}$
\end{tabular}
}

\tikzset{
aux/.style={
  align=left,
  draw=black,
  fill=white
  }
}

\newcommand{\examplefpi}[3]{
  $\ifpmanual{#1}{#2}{\fp{#3}}$
}
\newcommand{\exampleiis}[4]{
\begin{tabular}{ c | c  | c }
{#1} & {#3} & {#2} \\ 
\end{tabular} $_{(#4)}$
}

\tikzset{
fpi/.style={
  align=left,
  draw=black,
  fill=white
  }
}

\tikzset{
iis/.style={
  align=left,
  draw=black,
  fill=white
  }
}

\tikzset{
local/.style={
  }
}

\tikzstyle{edge} = [draw,thick,opacity=0.25]
\tikzstyle{dep} = [draw,thick]

\tikzset{
skiplistnode/.style={
  align=center,
  draw=black,
  fill=white,
minimum width={width("\exampleg") + width("x")},
minimum height={height("\exampleh") + height("h"}
  }
}

\tikzset{
fingerprintnode/.style={
  align=center,
  draw=black,
  thick,
  fill=lightgray,
minimum width={width("\exampleg") + width("x")},
minimum height={height("\exampleh") + height("h"}
  }
}




\begin{document}
\maketitle

% !TEX root = main.tex

\thispagestyle{plain}
\begin{center}
    \Large
    \textbf{Efficient Synchronization of Recursively Partitionable Data Structures}
       
    \vspace{0.9cm}
    \textbf{Abstract}
\end{center}

Given two nodes in a distributed system, each of them holding a data structure, one or both of them might need to update their local replica based on the data available at the other node.
An efficient solution should avoid redundantly sending data to a node which already holds it.

We give conceptually simple yet asymptotically efficient probabilistic solutions based on recursively exchanging fingerprints for data structures of exponentially decreasing size, obtained by recursively partitioning the data structures.
We apply the technique to sets, maps and radix trees.
For data structures containing $n$ items, this leads to $\complexity{log(n)}$ round-trips.
We give a scheme by which the fingerprints can be computed in $\complexity{log(n)}$ time, based on an auxiliary data structure which requires $\complexity{n}$ space and which can be updated to reflect changes to the underlying data structure in $\complexity{log(n)}$ time.

To minimize the number of round-trips, the technique requires up to $\complexity{n}$ space per synchronization session.
It can be adapted to require only a bounded amount of memory, which is essential for robust, scalable implementations.
While this increases the worst-case number of round-trips, it guarantees continuous progress, even in adversarial environments.

\tableofcontents

\chapter{Introduction}
\label{introduction}
% !TEX root = ../main.tex

One of the problems that needs to be solved when designing a distributed system is how to efficiently synchronize data between nodes.
Two nodes may each hold a particular set of data, and may then wish to exchange the ideally minimum amount of information until they both reach the same state.
Typical ways in which this can happen are one node taking on the state of the other, or both nodes ending up with the union of information available between the two of them.

\section{Motivating Examples}

Distributed version control systems can be seen as an example of the latter case: users independently create new objects which describe changes to a directory, and when connecting to each other, they both fetch all new updates from the other in order to obtain a (more) complete version history.
Regarded more abstractly, the two nodes compute the union of the sets of objects they store.
A version control system might attempt to leverage structured information about those objects, such as a happened-before relation, but this does not lead to good worst-case guarantees.
The set reconciliation protocol we give guarantees the exchange to only take a number of rounds logarithmic in the number of objects.

A different example are peer-to-peer publish-subscribe systems such as
Secure Scuttlebutt~\cite{tarr2019secure}. A node in the system can subscribe to
any number of topics, and nodes continuously synchronize all topics they share
with other nodes they encounter on a randomized overlay network. Scuttlebutt achieves
efficient synchronization by enforcing a linear happened-before
relation between messages published to the same topic, i.e.~each message
is assigned a unique sequence number that is one greater than the
sequence number of the previous message. When two nodes share interest
in a topic, they exchange the greatest sequence number they have for
this topic, whichever node sent the greater one then knows which
messages the other is missing.

The price to pay for this efficient protocol is that concurrent publishing of new
messages to the same topic is forbidden, since it would lead to
different messages with the same sequence number, breaking correctness
of the synchronization procedure. An unordered pubsub mechanism based on
set reconciliation would be able to support concurrent publishing, the
other design aspects such as the overlay network could be left
unchanged.

An example from a less decentralized setting are incremental software
updates. A server might host a new version of an operating system, users
running an old version want to efficiently download the changes. An
almost identical problem is that of efficiently creating a backup of a file
system to a server already holding an older backup. Both of these
examples can abstractly be regarded as updating a map from file paths to
file contents. Our protocol for mirroring maps could be used for
determining which files need to be updated. The protocol allows
synchronizing the actual files via an arbitrary nested synchronization
protocol, e.g.~rsync~\cite{tridgell1996rsync}.

\section{Efficiency Criteria}\label{efficiency-criteria}

There are a variety of criteria by which to evaluate a synchronization
protocol. We exemplify them by the trivial synchronization protocol,
which consists of both node immediately sending all their data to the
other node.

Let $n$ be the number of items held by node $\mathcal{A}$, and $m$ the number of items
held by node $\mathcal{B}$. To simplify things we assume for now that all items have
the same size in bytes.

The most obvious efficiency criteria are the \defined{total bandwidth}
($\complexity{n + m}$ bytes for the trivial protocol) and the number of
\defined{round-trips} ($1 \in \complexity{1}$ for the trivial protocol).

Efficiently using the network is not everything, the computational
\defined{time complexity per round-trip} must be feasible so that
computers can actually run the protocol. It is lower-bounded by the
amount of bytes sent in a given round, for the trivial protocol it is
$\complexity{n + m}$.

Similarly, the \defined{space complexity per round-trip} plays a relevant
role, since computers have only a limited amount of memory. In
particular, if an adversarial node can make a node run out of memory,
the protocol can only be run in trusted environments. Even then, when
non-malicious nodes of vastly differing computational capabilities
interact (e.g.~a microcontroller connecting to a server farm),
``accidental denial of service attacks'' can easily occur. Since the
trivial protocol does not perform any actual computation, its space
complexity per round-trip is in $\complexity{1}$.

The \defined{space complexity per session} measures the amount of state
nodes need to store across an entire synchronization session, in particular
while idly waiting for a response.

In addition to the space required for per-round-trip computations and per session, an
implementation of a protocol might need to store auxiliary information
that is kept in sync with the data to be synchronized, in order to
achieve sufficiently efficient time and space complexity per round-trip.
Of interest is not only the \defined{space for the auxiliary
information}, but also its \defined{update complexity} for keeping it
synchronized with the underlying data.

A protocol might be asymmetric, with different resource usage for different nodes.
If there is client and a clear server role, traditionally protocol designs aim to
keep the resource usage of the server as low as possible, motivated by the assumption
that many clients might concurrently connect to a single server, but a single client rarely
connects to a prohibitive amount of servers at the same time.

Any protocol design has to settle on certain trade-offs between these
different criteria, which will make it suitable for certain use cases,
but unsuitable for others. We do believe that our designs occupy a
useful place in the design space that is applicable to many relevant
problems, such as those mentioned in the introduction.

A final, ``soft'' criterium is that of simplicity. While ultimately time
and space complexities should guide adoption decisions, complicated
designs are often a good indicator that the protocol will never see any
deployment. Our designs require merely comparisons of (sums of) hashes,
and the auxiliary data structure that enables efficient implementation
is a simple balanced tree.

\section{Recursively Comparing
Fingerprints}\label{recursively-comparing-fingerprints}

We conclude the introduction with a brief sketch of a set reconciliation
protocol (i.e.~a protocol for computing the union of two sets on
different machines) that exemplifies the core ideas. The protocol
leverages the fact that sets can be partitioned into a number of smaller
subsets. The protocol assumes that the sets contain elements from
a universe on which there is a total order based on which intervals can be
defined, and that nodes can compute fingerprints for any subset of the
universe.

Suppose for example two nodes $\mathcal{A}$, $\mathcal{B}$ each hold a set of natural numbers.
They can reconcile all numbers within an interval as follows: $\mathcal{A}$ computes
a fingerprint over all the numbers it holds within the interval and then sends this
fingerprint to $\mathcal{B}$, together with the interval boundaries. $\mathcal{B}$ then computes
the fingerprint over all numbers it holds within that same interval. There
are three possible cases:

\begin{itemize}
\item
  $\mathcal{B}$ computed the same fingerprint it received, then the interval has
  been fully reconciled and the protocol terminates.
\item
  $\mathcal{B}$ has no numbers within the interval, $\mathcal{B}$ then notifies $\mathcal{A}$, $\mathcal{A}$
  transmits all its numbers from the interval, and the interval has been
  fully reconciled.
\item
  Otherwise, $\mathcal{B}$ splits the interval into two sub-intervals, such that $\mathcal{B}$
  has a roughly equal number of numbers within each interval. $\mathcal{B}$ then
  initiates reconciliation for both of these intervals, the roles $\mathcal{A}$ and
  $\mathcal{B}$ reverse.
\end{itemize}

Crucially, in the last case, the two recursive protocol invocations can
be performed in parallel. The number of parallel sessions increases
exponentially, so the original interval is being reconciled in a number
of rounds logarithmic in the greater number of items held by any node
within that interval.

\section{Thesis Outline}

The remainder of this thesis fleshes out details and applies the same
idea to some data structures, all of which share the property that they can be partitioned into smaller instances of the same data structure.

The viability of this approach
hinges on the efficient computation of fingerprints, which is discussed
and solved in \cref{fingerprints}. We then give a thorough definition of the set
conciliation protocol in \cref{basic-set-reconciliation}, and prove its correctness and its
complexity guarantees. \Cref{bounded-set-reconciliation} gives a more concrete protocol that
allows nodes to enforce limits on the amount of computational resources
they spend, at the cost of increasing the number of roundtrips if these
resource limits are reached. \Cref{other-data-structures} shows how to apply the same basic
ideas to k-d-trees, maps, tries and radix trees (TODO update as this cristallizes), and briefly discusses
why it does not make sense to apply it to arrays. \Cref{related-work}
gives an overview of related work and justifies the chosen approach. We
conclude in \cref{conclusion}.

\chapter{Computing Fingerprints}
\label{fingerprints}
% !TEX root = ../main.tex

The protocols we will discuss work by recursively computing and comparing fingerprints of sets.
This chapter defines and motivates a specific fingerprinting scheme that admits efficient computation with small overhead for the storage and maintenance of auxiliary data structures. \Cref{initial-considerations} outlines the solution space and theoretic bounds. We examine randomized solutions that with high probability compute fingerprints in logarithmic time in \cref{randomization}. \Cref{group-fingerprints} characterizes a family of functions that admit efficient incremental computation, and \Cref{collisions} proposes members of this family that can be used for fingerprinting, and \cref{crypto} examine security concerns in the face of malicious parties trying to find fingerprint collisions.

%TODO: move the following definitions to where they are needed
%
%\begin{definition}
%Let $(U_0, \groupaddsym_0, \neutraladd_0)$ and $(U_1, \groupaddsym_1, \neutraladd_1)$ be monoids and let $fun{f}{U_0}{U_1}$. We call $\f$ a \defined{monoid homomorphism} if it satisfies two properties:
%
%\begin{description}
%  \item[preserves operation:] for all $x, y \in U_0$: $\f(x \groupaddsym_0 y) = \f(x) \groupaddsym_1 \f(y)$
%  \item[preserves neutral element:] $f(\neutraladd_0) = \neutraladd_1$
%\end{description}
%\end{definition}
%
%\begin{definition}
%Let $(U_0, \groupaddsym_0, \neutraladd_0)$ and $(U_1, \groupaddsym_1, \neutraladd_1)$ be monoids. The \defined{direct product of $(U_0, \groupaddsym_0, \neutraladd_0)$ and $(U_1, \groupaddsym_1, \neutraladd_)$} is the monoid $(U_0 \times U_1, \f, (\neutraladd_0, \neutraladd_1))$ with $\f((u_0, u_1), (v_0, v_1)) \defeq (u_0 \groupaddsym_0 v_0, u_1 \groupaddsym_1 v_1)$.
%\end{definition}

\section{Initial Considerations}
\label{initial-considerations}

Our protocols work by recursively testing fingerprints for equality. For our purposes, we can define a fingerprint or hash function as follows:

\begin{definition}
A \defined{hash function} is a function $\fun{\h}{U}{D}$ with a finite codomain such that for randomly chosen $u \in U$ and $d \in D$ the probability that $\h(u) = d$ is roughly\footnote{To keep the focus on data structure synchronization rather than being sidetracked by cryptography, we will for the most part keep arguments about probabilities qualitative rather than quantitative.} $\frac{1}{\abs{D}}$. $\h(u)$ is called the \defined{hash of $u$}, \defined{fingerprint of $u$} or \defined{digest of $u$}.
\end{definition}

Given a universe $U$ of items, a function $\enc : U \rightarrow \{0, 1\}^{*}$ for encoding items as binary strings, a linear order $\preceq$ on $U$, a hash function $\h: \{0, 1\}^{*} \rightarrow \{0, 1\}^{k}$ mapping binary strings to binary strings of length $k$ and some finite $S \subseteq U$, a natural starting point for defining a fingerprint of the set $S$ is to sort the items according to $\preceq$, concatenate the encodings, and hash the resulting string.

While this is straightforward to specify and implement, it does not suffice for our purposes. To allow for efficient set reconciliation, we need to be able to efficiently compute the new fingerprint after a small modification of the set such as insertion or deletion of a single item. Furthermore, we want to be able to efficiently compute the fingerprints of all subsets defined by an interval of the original set.

The fingerprint based on concatenating encodings does not allow for efficient incremental reevaluation. When an item is added to $S$ that is less than any item previously in $S$, the hash function needs to be run over the whole string of length $\complexity{\abs{S} + 1}$ again. Furthermore, for any subinterval of the set, a full fingerprint computation needs to be performed as well. Precomputing the fingerprints of all subintervals requires a prohibitive amount of space. Every subinterval corresponds to a substring of the string consisting of all items in $S$ in ascending order, so there are $\frac{\abs{S} \cdot (\abs{S} + 1)}{2} + 1 \in \complexity{n^2}$ many in total.

The go-to approach for efficiently handling small changes to a set of totally ordered items are (balanced) search trees, we briefly state some definitions.

\begin{definition}
Let $U$ be a set and $\preceq$ a binary relation on $U$.
We call $\preceq$ a \defined{linear order on $U$} if it satisfies three properties:

  \begin{description}
    \item[anti-symmetry:] for all $x, y \in U$: if $x \preceq y$ and $y \preceq x$ then $x = y$
    \item[transitivity:] for all $x, y, z \in U$: if $x \preceq y$ and $y \preceq z$ then $x \preceq z$
    \item[linearity:] for all $x, y \in U$: $x \preceq y$ or $y \preceq x$
  \end{description}

If $\preceq$ is a linear order, we write $x \prec y$ to denote that $x \preceq y$ and $x \neq y$.
\end{definition}

\begin{definition}
Let $U$ be a set, $\preceq$ a linear order on $U$, and $V \subseteq U$. Let $T$ be a rooted directed tree with vertex set $V$.

Let $v \in V$, then $T_v$ denotes the subtree of $T$ with root $v$.

$T$ is a \defined{binary search tree on V} if for all inner vertices $v$ with left child $a$ and right child $b$: $a' \prec v$ for all $a' \in T_a$ and $v \prec b'$ for all $b' \in T_b$.


\end{definition}

\begin{definition}
Let $T = (V, E)$ be a binary search tree and $\epsilon \in \mathbb{R}_{> 0}$.
We call $T$ \defined{$\epsilon$-balanced} if $\textit{height}(T) \leq \ceil*{\epsilon \cdot log_2(|V|)}$.
Since the precise choice of $\epsilon$ will not matter for our complexity analyses, we will usually simply talk about \defined{balanced} trees.
\end{definition}

In the context of fingerprinting, balanced trees often take the form of Merkle trees~\cite{merkle1989certified}, binary trees storing items in their leaves, in which each leaf vertex is labeled with the hash of the associated item, and inner vertices are labeled with the hash of the concatenation of the child labels. The root label serves as a fingerprint for the set of items stored in the leaves.

When inserting or removing an item, the number of labels that need updating is proportional to the length of of the path from the root to the item, so in a balanced tree of $n$ items $\complexity{\mathit{log}(n)}$. The problem with this approach however is that fingerprints are no longer unique: there are different balanced search trees storing the same items set, and different tree arrangements result in different root labels.

Unfortunately it does not suffice to specify a particular balancing scheme, since different insertion orders of the same overall set of items can result in different trees, even when using the same balancing scheme. While this is sufficient for a setting in which only a single machine updates the set and all other machines apply updates in the same order, as assumed e.g. in~\cite{nissim1998certificate}, we aim for a less restrictive setting in which the evolution of the local set does not influence synchronization.

An alternative would be to define exactly one valid tree shape for any set of items, but this precludes logarithmic time complexity of updates, as \cite{uniquerepresentation} shows that search, insertion and deletion in such trees take at least $\complexity{\sqrt{n}}$ time in the worst case.

We will inspect two options for cheating this lower bound and to achieve logarithmic complexity: utilizing randomization to define a unique tree layout which allows logarithmic operations with high probability, which we examine in \cref{randomization}, or letting the fingerprint function abstract over the tree shape, downgrading it to an implementation detail, which we examine in \cref{group-fingerprints} and beyond.

\section{Pseudorandom Data Structures}
\label{randomization}

TODO: hash tries, possibly skip lists

% tarjan set equality testing https://apps.dtic.mil/sti/pdfs/ADA223643.pdf
% pugh function caching data structures http://matthewhammer.org/courses/csci7000-s17/readings/Pugh89.pdf

\section{Incremental Computations}
\label{group-fingerprints}

We now study a family of fingerprinting functions for sets that admit efficient incremental computation based on an auxiliary tree structure, but whose output does not depend on the exact shape of that tree. We first examine a general class of such functions, which reduce a finite set to a single value according to a monoid.

\begin{definition}
Let $M$ be a set, $\groupaddsym: M \times M \rightarrow U$, and $\neutraladd \in M$.

We call $(U, \groupaddsym, \neutraladd)$ a \defined{monoid} if it satisfies two properties:

  \begin{description}
    \item[associativity:] for all $x, y, z \in M$: $\groupadd{(\groupadd{x}{y})}{z} = \groupadd{x}{\groupadd{y}{z}}$
    \item[neutral element:] for all $x \in M$: $\groupadd{\neutraladd}{x} = x = \groupadd{x}{\neutraladd}$.
  \end{description}
\end{definition}

\begin{definition}
\label{def-lift}
Let $U$ be a set, $\preceq$ a linear order on $U$, $\mathcal{M} \defeq (M, \groupaddsym, \neutraladd)$ a monoid, and $\fun{\f}{U}{M}$.

We \defined{lift $\f$ to finite sets via $\mathcal{M}$} to obtain $\fun{\lift{\f}{\mathcal{M}}}{\powerset{U}}{M}$ with:

\begin{align*}
\lift{\f}{\mathcal{M}}(\emptyset) &\defeq \mymathbb{0}\\
\lift{\f}{\mathcal{M}}(S) &\defeq \f(\min_{\preceq}(S)) \oplus \lift{\f}{\mathcal{U}}(S \setminus \min_{\preceq}(S))\\
\end{align*}

In other words, if $S = \set{s_0, s_1, \ldots, s_{\abs{S} - 1}}$ with $s_0 \prec s_1 \prec \ldots \prec s_{\abs{S} - 1}$, then $\lift{\f}{\mathcal{M}}(S) = \groupadd{\f(s_0)}{\groupadd{\f(s_1)}{\groupadd{\ldots}{\f(s_{\abs{S} - 1})}}}$.
\end{definition}

Functions of the form $\lift{\f}{\mathcal{M}}$ can be incrementally computed by using labeled binary search trees:

\begin{definition}
Let $U$ be a set, $S \subset U$ a finite set, $\preceq$ a linear order on $U$, $\mathcal{M} \defeq (M, \groupaddsym, \neutraladd)$ a monoid, $\fun{\f}{U}{M}$, and let $T$ be a binary search tree on $S$.

We define a \defined{labeling function} $\fun{\liftlabel{\f}{\mathcal{M}}}{S}{M}$:

  \[
   \liftlabel{\f}{\mathcal{M}}(v) \defeq \begin{cases}
\f(v), &  \text{for leaf $v$} \\
\groupadd{\liftlabel{\f}{\mathcal{M}}(c_{<})}{\f(v)} & \, \text{v internal vertex with left child $c_{<}$ and no right child}\\
\groupadd{\f(v)}{\liftlabel{\f}{\mathcal{M}}(c_{<})} & \, \text{v internal vertex with right child $c_{>}$ and no left child}\\
\groupadd{\liftlabel{\f}{\mathcal{M}}(c_{<})}{\groupadd{\f(v)}{\liftlabel{\f}{\mathcal{M}}(c_{>})}} & \, \text{v internal vertex with left child $c_{<}$ and right child $c_{>}$}
\end{cases}
  \]

TODO fix overflow
See \cref{fig:fp-tree} for an example.
\end{definition}

\begin{figure*}
\begin{scaletikzpicturetowidth}{\textwidth}
\begin{tikzpicture}[scale=\tikzscale]
	\pgfdeclarelayer{background}
	\pgfdeclarelayer{foreground}
	\pgfsetlayers{background,main,foreground}
	
	\begin{pgfonlayer}{main}
		%vertices
		\node (vroot) at (0, 1) [aux] {\aux{\exampled}{
\groupadd{
    (\groupadd{\hexamplea}{\groupadd{\hexampleb}{\hexamplec)}}}
{\groupadd{\hexampled}
{(\groupadd{\hexamplee}{\groupadd{\hexamplef}{\hexampleg}})}
}
}};

		\node (v00) at (-4, -1) [aux] {\aux{\exampleb}{\groupadd{\hexamplea}{\groupadd{\hexampleb}{\hexamplec}}}};
		\node (v01) at (4, -1) [aux] {\aux{\examplef}{\groupadd{\hexamplee}{\groupadd{\hexamplef}{\hexampleg}}}};

                \node (v10) at (-6, -3) [aux] {\aux{\examplea}{\hexamplea}};
                \node (v11) at (-2, -3) [aux] {\aux{\examplec}{\hexamplec}};
                \node (v12) at (2, -3) [aux] {\aux{\examplee}{\hexamplee}};
                \node (v13) at (6, -3) [aux] {\aux{\exampleg}{\hexampleg}};
		%edges
                \draw (vroot) edge[edge] (v00);
                \draw (vroot) edge[edge] (v01);

		\draw (v00) edge[edge] (v10);
		\draw (v00) edge[edge] (v11);
		\draw (v01) edge[edge] (v12);
		\draw (v01) edge[edge] (v13);
	\end{pgfonlayer}
\end{tikzpicture}
\end{scaletikzpicturetowidth}

\caption{
A balanced search tree labeled by $\liftlabel{\h}{(M, \groupaddsym, \neutraladd)}$. For fingerprinting, $\h$ could be a hash function and $\groupaddsym$ the xor operation on fixed-width bitstrings.
}

\label{fig:fp-tree}
\end{figure*}

\begin{proposition}
Let $U$ be a set, $S \subset U$ a finite set, $\preceq$ a linear order on $U$, $\mathcal{M} \defeq (M, \groupaddsym, \neutraladd)$ a monoid, $\fun{\f}{U}{M}$, and let $T$ be a binary search tree on $S$ with root $r \in S$.

Then $\liftlabel{\f}{\mathcal{M}}(r) = \lift{\f}{\mathcal{M}}(S)$.

\begin{proof}
By induction on the number of vertices of $T$.

\textbf{IB:} If $r$ is a leaf, then $\abs{\V(T)} = 1$ and thus $\liftlabel{\f}{\mathcal{M}}(r) \overset{\text{def}}= \f(r) \overset{\text{def}}= \lift{\f}{\mathcal{M}}(\V(T)) = \lift{\f}{\mathcal{M}}(S)$.

\textbf{IH:} Let  $c_{<}$ and $c_{>}$ be vertices for which $\liftlabel{\f}{\mathcal{M}}(c_{<}) = \lift{\f}{\mathcal{M}}(\V(T_{c_{<}}))$ and $\liftlabel{\f}{\mathcal{M}}(c_{>}) = \lift{\f}{\mathcal{M}}(\V(T_{c_{>}}))$.

\textbf{IS:} If $r$ is an internal vertex with left child $c_{<}$ and right child $c_{>}$, then:

\begin{align*}
 \liftlabel{\f}{\mathcal{M}}(r) &\overset{\text{def}}= \groupadd{\liftlabel{\f}{\mathcal{M}}(c_{<})}{\groupadd{\f(r)}{\liftlabel{\f}{\mathcal{M}}(c_{>})}}\\
&\overset{\text{IH}}= \groupadd{\lift{\f}{\mathcal{M}}(\V(T_{c_{<}}))}{\groupadd{\f(p)}{\lift{\f}{\mathcal{M}}(\V(T_{c_{>}}))}}\\
& \overset{\text{def}}= \lift{\f}{\mathcal{M}}(\V(T))\\
&= \lift{\f}{\mathcal{M}}(S)\\
\end{align*}

The cases for internal vertices with exactly one child follow analogously.
\end{proof}
\end{proposition}

This correspondence can be used to incrementally compute $\lift{\f}{\mathcal{M}}(S)$: Initially, a labeled search tree storing the items in $S$ is constructed. $\lift{\f}{\mathcal{M}}(S)$ is the root label. When an item is inserted or removed, only the labels on the path from the root to the point of modification require recomputation, so only a logarithmic number of operations is performed if a self-balancing tree is used.

Note that the exact shape of the tree determines the grouping of how to apply $\groupaddsym$, but by associativity all groupings yield the same result. All trees storing the same set have the same root label.

If $U$ is small enough that space usage of $\complexity{\abs{U}}$ is acceptable, an implicit tree representation such as a binary indexed tree (Fenwick tree)~\cite{fenwick1994new} can be used. Array positions that correspond to some $u \in U \setminus S$ are simply filled with a dummy value whose hash is defined to be $\neutraladd$.

\subsection{Subsets}

In addition to incremental computation of the fingerprint of a given set, the reconciliation protocol also requires the efficient computation of the fingerprints of arbitrary intervals of the given set. We first fix some terminology and notation:

\begin{definition}
\label{def-interval}
Let $S \subseteq U$, $\preceq$ a linear order over $U$, and $x, y \in U$.

The \defined{interval from $x$ to $y$ in $S$}, denoted by $\interval{x}{y}{S}$, is the set $\set{s \in S \mid x \preceq s \prec y}$ if $x \prec y$, and $S \setminus \interval{y}{x}{S}$ if $y \preceq x$. We call $x$ the \defined{lower boundary} and $y$ the \defined{upper boundary} of the interval (even if $y \preceq x$).

Note that the upper boundary is excluded from the interval, so in particular $\interval{x}{x}{S} = S$.
\end{definition}

In the remainder of this section, we assume that for all intervals $\interval{x}{y}{S}$ we have $x \prec y$. If that is not the case, all computations can be performed for the sets $\set{s \in S \mid x \preceq s}$ and $\set{s \in S \mid s \prec y}$ which partition $\interval{x}{y}{S}$ if $y \preceq x$, and the resulting values can be combined via $\groupaddsym$ to obtain the desired result.

Given a balanced search tree $T$ with root $r$ for a set $S$ that is labeled by $\liftlabel{\f}{\mathcal{M}}$, we can compute $\lift{\f}{\mathcal{M}}(\interval{x}{y}{S})$ in logarithmic time. Intuitively, one traces paths in $T$ to both $x$ and $y$, and then the result is the sum over all vertices ``in the area between'' these paths. For every vertex on the traced paths, the label of the ``inner'' child vertex summarizes multiple vertices within the area. Summing over all these children yields the value corresponding to the whole inner area. Since the length of the delimiting paths is logarithmic, overall only a logarithmic number of labels needs to be added up.

\Cref{listing:subset-fingerprint} gives a precise definition of how to compute $\lift{\f}{\mathcal{M}}(\interval{x}{y}{S})$ as Haskell~98~\cite{jones2003haskell} code, for clarity of presentation only complete binary trees are considered. Arbitrary binary trees can also have inner nodes with exactly one child. These can be handled with almost the same algorithm by acting as if these nodes had a second child labeled with $\neutraladd$.

\begin{listing}
\begin{minted}[linenos, mathescape]{haskell}
-- $\mathtt{U}$ is the type of items, $\mathtt{D}$ the type of fingerprints.
-- A node is either a leaf or an inner vertex.
data Node = Leaf U D | Inner Node U Node D

-- Extracts the label of a node.
label :: Node -> D
label Leaf _ fp      = fp
label Inner _ _ _ fp = fp

-- Compute the fingerprint over all items stored in $v$
-- within the interval $\interval{x}{y}{S}$, assuming $x \prec y$.
intervalFingerprint :: Node -> U -> U -> D
intervalFingerprint v x y = case findInitial v x y of
    Nothing               -> 0
    Just (Leaf _ fp)      -> fp
    Just (Inner l v' r _) -> (sumGeq l x) + (f v') + (sumLt r y)
                                           
-- Find the node within $\interval{x}{y}{S}$ that is closest to the root,
-- assuming $x \prec y$.
findInitial :: Node -> U -> U -> Maybe Node
findInitial (Leaf v _) x y
    | x <= v && v < y = Just v
    | otherwise       = Nothing
findInitial (Inner l v r _) x y
    | v < x           = findInitial r x y
    | v >= y          = findInitial l x y
    | otherwise       = Just v

-- Sum up the fingerprints of all items in the given tree
-- which are greater than or equal to $x$.
sumGeq :: Node -> U -> D
sumGeq (Leaf v fp) x
    | v < x     = 0
    | otherwise = fp
sumGeq (Inner l v r _) x
    | v < x     = sumGeq r x
    | otherwise = (sumGeq l x) + (f v) + (label r)

-- Sum up the fingerprints of all items in the given tree
-- which are strictly less than $y$.
sumLt :: Node -> U -> D
sumLt (Leaf v fp) y
    | v >= y    = 0
    | otherwise = fp
sumLt (Inner l v r _) y
    | v >= y    = sumLt l y
    | otherwise = (label l) + (f v) + (sumLt r y)
\end{minted}
\caption{Computing $\interval{x}{y}{S}$ from the complete labeled search tree of $S$.}
\label{listing:subset-fingerprint}
\end{listing}

The algorithm proceeds by first finding the vertex $v$ with the smallest distance to the root such that $x \preceq v \prec y$. This might be $r$ itself. If there is no such $v$, then $\interval{x}{y}{S} = \emptyset$ and thus $\lift{\f}{\mathcal{M}}(\interval{x}{y}{S}) = \neutraladd$. All vertices of $T$ that are not vertices in $T_v$ are either greater than or equal to $y$ if $v \prec x$, or strictly less than $x$ if $x \prec v$, in either case they do not influence $\lift{\f}{\mathcal{M}}(\interval{x}{y}{S})$.

If $v$ is a leaf, $\interval{x}{y}{S} = \set{v}$ and thus $\lift{\f}{\mathcal{M}}(\interval{x}{y}{S}) = \f(v)$. Otherwise, let $c_{<}$ be the left child of $v$ and let $c_{>}$ be the right child. Since all vertices in $T_{c_{>}}$ are greater than $v$, they are in particular greater than $x$. Analogously all vertices in $T_{c_{<}}$ are less than $y$.

Keeping in mind that $\interval{x}{\max(\V(T_{c_{<}}))}{\V(T_{c_{<}})}$ simply denotes the set of vertices in $T_{c_{<}}$ that are strictly greater than $x$, and analogously $\interval{\min(\V(T_{c_{>}}))}{y}{\V(T_{c_{>}})}$ denotes the vertices of $T_{c_{>}}$ that are less than or equal to $y$, we have:

\begin{align*}
\interval{x}{y}{S} &= \disjointunion{\interval{x}{\max(\V(T_{c_{<}}))}{\V(T_{c_{<}})}}{\disjointunion{\set{v}}{\interval{\min(\V(T_{c_{>}}))}{y}{\V(T_{c_{>}})}}}\\
&\mathrm{implying}\\
\lift{\f}{\mathcal{M}}(\interval{x}{y}{S}) &= \groupadd{\lift{\f}{\mathcal{M}}(\interval{x}{\max(\V(T_{c_{<}}))}{\V(T_{c_{<}})})}{\groupadd{\f(v)}{\lift{\f}{\mathcal{M}}(\interval{\min(\V(T_{c_{>}}))}{y}{\V(T_{c_{>}})})}}\\
\end{align*}

Proving that \texttt{sumGeq} from \cref{listing:subset-fingerprint} does indeed sum over all $\f(v)$ in the given tree with $x \preceq v$, i.e. computes $\lift{\f}{\mathcal{M}}(\interval{x}{\max(\V(T_{c_{<}}))}{\V(T_{c_{<}})})$ can be done by a rather technical but straightforward induction which we omit, same for \texttt{sumLt} computing $\lift{\f}{\mathcal{M}}(\interval{\min(\V(T_{c_{>}}))}{y}{\V(T_{c_{>}})})$.

From those facts, correctness of \texttt{intervalFingerprintGroup} follows by the previous arguments.

The worst-case running time occurs when the vertex $v$ with the smallest distance to $r$ such that $x \preceq v \prec y$ is $r$ itself, since then both \texttt{sumGeq} and \texttt{sumLt} perform a traversal to a leaf of maximal length. Since $T$ is balanced, only a logarithmic number of recursive calls is executed. Assuming $\f$ can be computed in $\complexity{1}$, the resulting time complexity is in $\complexity{\log(\abs{S})}$.

A slightly simpler approach can be taken if $M$ has efficiently computable inverses with respect to $\groupaddsym$, i.e. if $(M, \groupaddsym, \neutraladd)$ is a group.

\begin{definition}
Let $(M, \groupaddsym, \neutraladd)$ be a monoid.
We call it a \defined{group} if for all $x \in M$ there exists $y \in M$ such that $\groupadd{x}{y} = \neutraladd$.
This $y$ is necessarily unique and denoted by $\inverseadd{x}$.
For $x, y \in M$ we write $\groupsubtract{x}{y}$ as a shorthand for $\groupadd{x}{\inverseadd{y}}$.
\end{definition}

Observe that $\interval{x}{y}{S} = \interval{\min(S)}{y}{S} \setminus \interval{\min(S)}{x}{S}$, and thus also $\lift{\f}{\mathcal{M}}(\interval{x}{y}{S}) = \groupadd{\inverseadd{\lift{\f}{\mathcal{M}}(\interval{\min(S)}{x}{S})}}{\lift{\f}{\mathcal{M}}(\interval{\min(S)}{y}{S})}$. Reusing the definitions from \cref{listing:subset-fingerprint}, we get:

\begin{minted}{haskell}
intervalFingerprintGroup :: Node -> U -> U -> D
intervalFingerprintGroup v x y = -(sumLt v x) + (sumLt v y)
\end{minted}

Alternatively, a slightly more efficient version that still does not require \texttt{sumGeq}:

\begin{minted}{haskell}
intervalFingerprint :: Node -> U -> U -> D
intervalFingerprint v x y = case findInitial v x y of
    Nothing               -> 0
    Just (Leaf _ fp)      -> fp
    Just (Inner l v' r _) -> -(sumLt l x) + (f v') + (sumLt r y)
\end{minted}

\section{Monoidal Fingerprints}
\label{collisions}

Now that we have characterized a family of functions that admit efficient recomputation in response to changes to the underlying set as well as efficient computation for intervals within the set, the remaining task is to find such functions which are also suitable fingerprints. This consists of deciding on the monoid of fingerprints, and choosing the mapping from items to monoid elements.

As the fingerprint of a singleton set $\lift{\f}{\mathcal{M}}(\set{u})$ is equal to $\f(u)$, $\f$ must itself already be a hash function. Typical hash functions map values to bit strings of a certain length, i.e. the codomain is $\set{0, 1}^k$ for some $k \in \N$. We will thus consider monoids whose elements can be represented by such bit strings.

A natural choice of the monoid universe is then $\interval{0}{2^k}{\N}$, some simple monoidical operations on this universe include bitwise xor, addition modulo $2^k$, and multiplication modulo $2^k$. In the following, addition and multiplication will always be implicitly taken modulo $2^k$. Note that $\xor$ and addition also admit inverses, so the slightly simplified computational fingerprints can be used.

Of these three options, multiplication is the least suitable, because multiplying $0$ by any number again yields $0$. Consequently for every set containing an item $u$ with $\f(u) = 0$ the fingerprint of the set would also be $0$, which clearly violates the criterion that all possible values for fingerprints occur with equal probability.

Addition and xor however are particularly well-behaved in that regard, as they form finite commutative groups:

\begin{proposition}
Let $\mathcal{G} \defeq (G, \groupaddsym, \neutraladd)$ be a finite commutative group, i.e. a group with a finite universe such that for all $x, y \in G$: $\groupadd{x}{y} = \groupadd{y}{x}$. Let $U$ be a set and let $\fun{f}{U}{G}$ be a hash function.

Then $\lift{\f}{\mathcal{G}}$ is a hash function as well.

\begin{proof}
We first show that for randomly chosen $x, y, z \in G$ the probability that $\groupadd{x}{y} = z$ is $\frac{1}{\abs{G}}$.

For $x, z \in G$ there is $y \in G$ such that $\groupadd{x}{y} = z$, namely $y \defeq \groupsubtract{z}{x}$ (because $\groupadd{x}{\groupsubtract{z}{x}} \overset{\text{commutativity}}= \groupadd{\groupadd{x}{\inverseadd{x}}}{z} = z$). As $G$ is finite, this $y$ has to be unique, since otherwise that would not be enough elements left that can be added to $x$ to result in all of the $\abs{G} - 1$ possible remaining $z'$. Thus for any fixed $x, z$ the probability that a randomly chosen $y$ satisfies $\groupadd{x}{y} = z$ is $\frac{1}{\abs{G}}$.

Computing $\lift{\f}{\mathcal{G}}(S)$ consists of repeatedly adding group elements which by themselves are distributed uniformly at random if $S$ was chosen randomly and $\f$ is a high quality hash function. Thus the accumulated value after every step is any given $z \in G$ with probability $\frac{1}{\abs{G}}$, as is in particular the probability for the final result being equal to $z$.
\end{proof}

A more formal proof for xor specifically is given in section 6.2 of \cite{maziarz2021hashing}.
\end{proposition}

By the same argument, knowing the fingerprint for some set does not provide any information about the fingerprints for sets that differ by even only a single value. In conclusion, high quality fingerprints can be achieved by choosing any transitive and commutative operation for the monoid, for example xor or addition modulo $2^k$, as long as values are mapped into the monoid with a high quality hash function.

While multiplication does not form a group when performed on the numbers in $\interval{0}{2^k}{\N}$, there still are groups based on multiplication modulo some number, e.g. $\Z_n^\ast$, the group yielded by multiplication modulo $n$ on the set $\set{x \in \interval{0}{n}{\N} \mid \text{$x$ is coprime to $n$}}$. In the following, when talking about multiplication, we will assume that the universe is chosen such that multiplication forms a group.

\section{Cryptographically Secure Fingerprints}
\label{crypto}

In the protocols for synchronizing data structures, fingerprints of sets are used for probabilistic equality checking: sets with equal fingerprints are assumed to be equal. Synchronization can thus become faulty if it involves unequal sets with equal fingerprints. If the universe of possible fingerprints is chosen large enough, and the distribution of fingerprints of randomly chosen sets is random within that universe, the probability for this occurring becomes negligible.

Random distribution of input sets is however a very strong assumption. What if a malicious party can influence the sets to be fingerprinted, with the goal of causing fingerprint collisions and consequently triggering faulty behavior of the system? Cryptographically secure fingerprints are an answer to this problem, being chosen such that it is computationally infeasible for an adversary to find inputs that lead to faulty synchronization.

\subsection{General Considerations}

A typical definition of cryptographically secure hash functions is the following~\cite{menezes2018handbook}:

\begin{definition}
A \defined{secure hash function} is a hash function $\fun{\h}{U}{D}$ that satisfies three additional properties:

\begin{description}
  \item[pre-image resistance:] Given $d \in D$, it is computationally infeasible to find a $u \in U$ such that $\h(u) = d$.
  \item[second pre-image resistance:] Given $u \in U$ it is computationally infeasible to find a $u' \in U, u' \neq u$ such that $\h(u) = \h(u')$.
  \item[collision resistance:] It is computationally infeasible to find $u, v \in U, u ~= v$ such that $\h(u) = \h(v)$.
\end{description}
\end{definition}

Pre-image resistance has no influence on the vulnerability of the protocol to malicious actors, so all of the following discussion will focus on collision resistance only.

Since $\lift{\f}{\mathcal{M}}(\set{u}) = \f(u)$, $\f$ must necessarily be collision resistant if $\lift{\f}{\mathcal{M}}$ is to be collision resistant. This alone is unfortunately not sufficient, we will see a specific counterexamples in the following subsections. Choosing a secure hash function $\f$ always comes with a performance cost, insecure hash functions usually take less time and less space to compute. If the synchronization protocol is only being run in a trusted environment, an insecure hash function might be preferable.

Whether a hash function is secure is not a binary dichotomy, but depends on what is considered ``feasible'' for an adversary. Greater security can usually be obtained at the cost of longer digests and longer computation times. Before presenting options for secure hash functions, we thus examine the impact of hash collisions first.

We can generally distinguish between malicious actors in two different positions: those who can actively impact the contents of the data structure to be synchronized, and those who passively relay updates and need to search for a collision within the available data. As a set of size $n$ has $2^n$ subsets, if fingerprints are bit strings of length $k$, then by the pigeonhole principle a fingerprint collision can be found within any set of size at least $k + 1$.

An attack against the fingerprinting scheme by an active adversary can involve computing many fingerprints and adding the required items to the set once a collision has been found. Such an attack is not usable by the passive adversary. We will primarily focus on discussing active adversaries, as they are strictly more powerful than passive ones. Yet it should be kept in mind that passive adversaries can be more common in certain settings, particularly in peer-to-peer systems: if a node is interested in synchronizing a data structure, it probably trusts the source of the data, otherwise it would have little reason for expending resources on synchronization. The data may however be synchronized not with the original source but with completely untrusted nodes. Additionally, an active adversary that does not want to risk detection by adding suspicious items to the data structure is restricted to the operations of a passive adversary.

Fingerprint collisions result in parts of the data structure not being synchronized, so information is being withheld from one or both of the synchronizing nodes. When a malicious node synchronizes with an honest one, the malicious node can withhold arbitrary information by simply pretending not to have certain data, which does not require finding collisions at all.

So the cases in which a malicious node can do actual damage by finding a collision are those where it supplies data to two different, honest nodes such that these two nodes perform faulty synchronization amongst each other. Specifically: let $\mathcal{M}$ be a malicious node, $\mathcal{A}$ and $\mathcal{B}$ be honest nodes, then a successful attack consists of $\mathcal{M}$ crafting sets $X_A, X_B$ and sending these to $\mathcal{A}$ and $\mathcal{B}$ respectively, so that when $\mathcal{A}$ and $\mathcal{B}$ then run the synchronization protocol, they end up with different data structures. A passive adversary does not craft $X_A, X_B$ but must find them as subsets of some set $X$ supplied by an honest node. As we assume the underlying hash function $\f$ to be secure, at least one non-singleton set has to be involved in a collision.

There are some qualitative arguments that even if an adversary finds a fingerprint collision, the impact is rather low. Let $S_A \subseteq X_A$ and $S_B \subseteq X_B$ be nonequal sets with the same fingerprint. To have any impact on the correctness of a particular protocol run, their two fingerprints need to actually be compared during that run. For that to happen, they need to be of the form $\interval{x_A}{y_A}{X_A}$ and $\interval{x_B}{y_B}{X_B}$ respectively. The fingerprints of these intervals can then be compared if one of the nodes sends the fingerprint for the interval $\interval{\min(x_A, x_B)}{\max(y_A, y_B)}{X_i}$. That alone is still not sufficient, as any item within that interval that is not part of the sets would change the fingerprint. So the two intervals actually need to be of the form $\interval{\min(x_A, x_B)}{\max(y_A, y_B)}{X_A}$ and $\interval{\min(x_A, x_B)}{\max(y_A, y_B)}{X_B}$, or simplified: there have to be $x, y \in U$ such that $S_A = \interval{x}{y}{X_A}$ and $S_B = \interval{x}{y}{X_B}$.

If the adversary has found such sets, that is still no guarantee that the interval $\interval{x}{y}{X_i}$ is being compared during the synchronization session of $\mathcal{A}$ and $\mathcal{B}$. For a set containing $n$ items, there are $n^2 - (n - 1) \in \complexity{n^2}$ distinct intervals, but only $\complexity{n}$ are compared in a given protocol run, since the worst-case message complexity is $\complexity{n}$ (see \cref{communication-complexity}). These numbers should only be considered as rough guidelines, they gloss over details such as the fact that there are more intervals containing roughly $\frac{n}{2}$ items then there are intervals containing almost all or almost no items. Yet they demonstrate that finding suitable colliding intervals still does not guarantee that a particular pair of nodes will be affected. In particular, there is no need for $\mathcal{A}$ and $\mathcal{B}$ to choose the interval boundaries that occur in a protocol run deterministically (see \cref{random-boundaries}). They can even perform multiple randomized protocol runs in parallel, while keeping track of item transmissions and sending every item at most once across all these protocol runs.

Another factor mitigating the impact of an adversary finding fingerprint collisions is the communication with other, non-colluding nodes. A fourth party could send some $u \in U, x \preceq u \prec y$ to $\mathcal{A}$ or $\mathcal{B}$ before $\mathcal{A}$ and $\mathcal{B}$ synchronize, disrupting the collision.

Finally, in systems where nodes repeatedly synchronize with different other nodes, a single fingerprint collision in a single synchronization session would merely delay propagation of information rather than stop it completely (note that this does not hold for collisions of two singleton sets). Since the attack model requires both $\mathcal{A}$ and $\mathcal{B}$ to synchronize with more than one node in total, this might apply to many affected settings. Peer-to-peer systems communicating on a random overlay network in particular fall into this category. A malicious actor with enough control over the communication of other nodes to guarantee a tangible benefit from fingerprint collisions can likely disrupt operation of the network more effectively by exercising that control than by sabotaging synchronization.

All of these arguments are however purely qualitative and should as such be taken into account with caution, they are not a substitute for quantitative cryptographic analysis. A strong attacker might be able to find many pairs of sets of colliding fingerprints, or many sets that all share the same fingerprint, and none of the above arguments consider these cases.

In a system with a consensus mechanism among all participating nodes, the choice of $\f$ can periodically be changed, with the frequency being some function of the time it takes to find a viable collision, the cost of rebuilding all auxiliary data structures, and the general level of paranoia among the participating nodes. The $\complexity{n}$ cost of rebuilding the data structures is then being amortized over the number of synchronization sessions occurring between rebuilds, which is still an improvement over protocols that need to perform $\complexity{n}$ computations per synchronization session.

\subsection{Attacking Specific Monoids}

Multiplication, addition and xor as a way of combining hashes have been studied in \cite{bellare1997new}, in the context of hashing sequences, of which our ordered sets are a special case. Their setting requires not only associativity but also commutativity. They thoroughly break xor by reducing the problem of finding a collision to that of solving a system of linear equations. We will not restate the full attack, but we will describe a weaker but simpler mechanism based on similar ideas that allows finding subsets whose fingerprint is a specific target value. This can be used as an optimization in a trusted setting (see \cref{subset-checks}).

The main observation is that xor is the additive operation in $\mathds{F}_2$, the finite field on two elements $\set{0, 1}$. A fingerprint can be interpreted as a vector $\begin{bmatrix}b_1 & b_2 & \ldots & b_k\end{bmatrix} \in \set{0, 1}^{1 \times k}$. The fingerprint of a set $S = \set{s_1, s_2, \ldots, s_n}$ can thus be computed as the sum (within $\mathds{F}_2$, i.e. as the xor) of the vectors corresponding to $s_1, s_2, \ldots, s_n$. The fingerprint of some $S' \subseteq S$ can also be regarded as the sum over all vectors, but each vector being first multiplied with a coefficient of $1$ if $s_i \in S'$ and coefficient $0$ if $s_i \notin S'$. In other words, the fingerprint of $S'$ is a linear combination of the hashes of the items in $S$. 

This enables us to efficiently find subsets whose fingerprint are a particular vector. Let $S = \set{s_1, s_2, \ldots, s_n}$ be a set of items, and let $b = \begin{bmatrix}b_1 & b_2 & \ldots & b_k\end{bmatrix} \in \set{0, 1}^{1 \times k}$ be the target fingerprint. For $0 < i \leq n$ define $a_i = \begin{bmatrix}a_{i, 1} & a_{i, 2} & \ldots & a_{i, k}\end{bmatrix} \in \set{0, 1}^{1 \times k}$ to be $\f(s_i)$ interpreted as a vector over $\mathds{F}_2$. The coefficients for linear combinations of the $a_i$ equal to $b$ are the solutions to the following system of linear equations over $\mathds{F}_2$:

\[
\begin{bmatrix}
a_{1, 1} & a_{1, 2} & \ldots & a_{1, k}\\
a_{2, 1} & a_{2, 2} & \ldots & a_{2, k}\\
\vdots & \vdots & \ddots & \vdots \\
a_{n, 1} & a_{n, 2} & \ldots & a_{n, k}\\
\end{bmatrix} \cdot \begin{bmatrix}
x_1\\
x_2\\
\vdots\\
x_n
\end{bmatrix} = \begin{bmatrix}
b_1\\
b_2\\
\vdots\\
b_n
\end{bmatrix}
\]

Solutions can be found by gaussian elimination in $\complexity{n^3}$ time, an exponential improvement over brute forcing by computing the fingerprints of all subsets.

As xor admits polynomial-time attacks by solving linear equations, the authors of \cite{bellare1997new} next consider addition and multiplication. They unify parts of their discussion by relating the hardness of finding collisions to solving the balance problem: in a commutative group $(G, \groupaddsym, \neutraladd)$, given a set of group elements $S = \set{s_1, s_2, \ldots, s_n}$, find disjoint, nonempty subsets $S_0 = \set{s_{0, 0}, s_{0, 1}, \ldots, s_{0, k}} \subseteq S, S_1 = \set{s_{1, 0}, s_{1, 1}, \ldots, s_{1, l}} \subseteq S$ such that $s_{0, 0} \groupaddsym s_{0, 1}  \groupaddsym \ldots  \groupaddsym s_{0, k} = s_{1, 0}  \groupaddsym s_{1, 1}  \groupaddsym \ldots  \groupaddsym s_{1, l}$. They then reduce the hardness of the balance problem to other problems.

For addition, the balance problem is as hard as subset sum, which was at the time of publication conjectured to be sufficiently hard. Wagner showed however in \cite{wagner2002generalized} how to solve the balance problem in subexponential time for addition. To give an impression for the impact of this attack, consider \cite{mihajloska2015reviving} which uses addition for combining SHA-3~\cite{dworkin2015sha} digests, producing fingerprints of length between $2688$ and $4160$ or $6528$ to $16512$ bits to achieve security levels of $128$ or $256$ bit respectively against Wagner's attack. \cite{lyubashevsky2005parity} gives an improvement over Wagner's attack finding collisions in $\complexity{2^{n^\epsilon}}$ for arbitrary $\epsilon < 1$, further weakening addition as a choice of monoid operation.

For multiplication, the balance problem is as hard as the discrete logarithm problem in the group. This is a more ``traditional'' hardness assumption than subset sum, there are groups for which no efficient algorithm is known. The main drawback is that multiplication is less efficient to compute that addition. \cite{stanton2010fastad} includes a comparison between the performance of addition and multiplication for incremental hashing, the additive hash outperforms the multiplicative one by two orders of magnitude, even though it uses longer digests to account for Wagner's attack.

For our context in which fingerprints are frequently sent over the network, longer computation time might still be preferable over longer hashes. Fingerprints based on multiplication nevertheless need larger digests than traditional, non-incremental hash functions. \cite{maitin2017elliptic} suggests fingerprints of $3200$ bit to achieve $128$ bit security, and motivated by this uses binary elliptic curves as the underlying group to achieve more compact fingerprints - fingerprints of length $2k$ bits give $\complexity{2^k}$ security.

\cite{bellare1997new} also proposes a fourth monoid based on lattices. \cite{lewi2019securing} give a specific instantiation providing 200 bits of security with fingerprints of size $16 \cdot 1024 = 16384$ bit.

All of the preceeding monoid operations are commutative, which our approach to fingerprint computation does not require. A typical associative but not commutative operation is matrix multiplication. Study of a family of hash functions based on multiplication of invertible matrices was initiated in \cite{zemor1991hash}, whose security is related to solving hard graph problems on the Caley graph of the group. \cite{petit2011rubik} gives a good overview about the general principle and the security aspects of Caley hash functions.

While \cite{tillich1994hashing}, an improvement over the originally proposed scheme, has been successfully attacked in \cite{grassl2011cryptanalysis}\cite{petit2010preimages}, there are several modifications~\cite{petit2009graph}\cite{bromberg2017navigating}\cite{sosnovski2016cayley} for which no attacks are currently known, and \cite{mullan2016text} showed random self-reducibility for Caley hash functions.

Whereas the schemes based on commutative groups operate on two bitstrings at a time, the Caley hash functions operate on two individual bits at a time. Attacks thus assume that the manipulated bits can be freely chosen, using this to e.g. craft palindromic inputs. This may mean that such attacks are hard to apply to our setting where the input bits are produced in non-reorderable batches by another, cryptographically secure hash function. After finding a bit sequence that yields a Caley hash collision, an attacker still needs to find items for which the concatenation of their hashes is the desired it sequence.

Alternatively, one can forgo the additional hash function and apply the Caley hash function directly to the encodings of items. This simplifies the overall scheme and removes reliance on a second hash function. On the other hand, if one expects the additional hash function to be harder to attack than the Caley hash, then making the attacks against the Caley hash function less flexible might overall make attacks more difficult. Furthermore, Caley hash functions are usually slower than typical choices for the hash function from items to bitstrings, so using both can produce significant speedups if items have long encodings.

Aside from Caley hashes, we are not aware of any non-commutative monoids used for hashing. We would further like to note that any hashes based on groups still have more structure than required for our designs, as we don't need existence of inverse elements. Strictly speaking our designs do not even require a neutral element: the fingerprint of the empty set never needs to be exchanged in a protocol run, so we could have just as well defined fingerprints for all nonempty sets without relying on the neutral element to do so. The presentation we chose is merely more elegant. And finally, we do not require pre-image resistance for our fingerprints. Suitable hash functions could thus be located in a more general design space than studied in any literature we know of.


\chapter{Set Reconciliation}
\label{basic-set-reconciliation}
% !TEX root = ../main.tex

In this chapter, we consider the set reconciliation protocol sketched in the introduction in greater detail.
We define the protocol in \cref{set-reconciliation-def}, prove its correctness in \cref{set-reconciliation-simple-correct}, and do a complexity analysis in \cref{set-reconciliation-complexity}.  \Cref{set-reconciliation-simple-optimizations} lists some optimizations which do not change the asymptotic complexity but which avoid some unnecessary work. We conclude the chapter with an example application in \cref{set-reconciliation-graphs}, briefly describing how the protocol can be applied to the synchronization of the hash graphs that arise e.g. in the context of distributed version control systems such as git~\cite{chacon2014pro}.

\section{Recursive Set Reconciliation}
\label{set-reconciliation-def}

The set reconciliation protocol assumes that there is a set $U$, a linear order $\preceq$ on $U$, a node $\mathcal{X}_0$ locally holding some $X_0 \subseteq U$, and a node $\mathcal{X}_1$ locally holding $X_1 \subseteq U$.
$\mathcal{X}_0$ and $\mathcal{X}_1$ exchange messages, a message consists of an arbitrary number of \defined{interval fingerprints} and \defined{interval item set}.
An interval fingerprint is a triple $\ifp{x}{y}{X_i}$ for $x, y \in U$, an interval item set a four-tuple $\iis{x}{y}{S}{b}$ for $x, y \in U, S \subseteq \interval{x}{y}{X_i}, b \in \{0, 1\}$. $b$ indicates whether the interval item set is a response to a previous interval item set.

Recall that $\fp{A}$ denotes the fingerprint for $A \subseteq U$, and that $\interval{x}{y}{A} \defeq \{a \in A | x \preceq a \prec y\}$.

When a node $\mathcal{X}_i$ receives a message, it performs the following actions:

\begin{itemize}
  \item For every interval item set $\iis{x}{y}{S}{b}$ in the message, all items in $S$ are added to the locally stored set $X_i$. If $b = 0$, the node then adds the interval item set $\iis{x}{y}{\interval{x}{y}{X_i} \setminus S}{1}$ to the response, unless $\interval{x}{y}{X_i} \setminus S = \emptyset$.
  \item For every interval fingerprint $\ifp{x}{y}{X_j}$ in the message, it does one of following:
    \begin{caselist}
      \case[Equal Fingerprints] \label{def-fingerprint-eq} If $\fp{\interval{x}{y}{X_j}} = \fp{\interval{x}{y}{X_i}}$, nothing happens.
      \case[Recursion Anchor] \label{def-recursion-anchor} The node may add the interval item set $\iisnatural{x}{y}{X_i}{0}$ to the response. If $\abs{\interval{x}{y}{X_j}} \leq 1$, it must do so.
      \case[Recurse] \label{def-recurse} Otherwise, the node selects $m_0 = x \prec m_1 \prec \ldots \prec m_k = y \in U$, $k \geq 2$ such that among all $\interval{m_0}{m_1}{X_i}$ for  $0 \leq l < k$ at least two intervals are non-empty. For all $0 \leq l < k$ it adds either the interval fingerprint $\ifp{m_l}{m_{l + 1}}{X_i}$ or the interval item set $\iisnatural{m_l}{m_{l + 1}}{X_i}{0}$ to the response.
    \end{caselist}
  \item If the accumulated response is nonempty, it is sent to the other node. Otherwise, the protocol has terminated successfully.
\end{itemize}

To initiate reconciliation of an interval $[x, y)$, a node $\mathcal{X}_i$ sends a message containing solely the interval fingerprint $\ifp{x}{y}{X_i}$.

\Cref{simple-set-reconciliation-example} gives an example run of the protocol.

\begin{figure*}
$X_0 \defeq \{\exampleb, \examplec, \exampled, \examplee, \examplef, \exampleh \}$
\hfill
$X_1 \defeq \{\examplea, \examplee, \examplef, \exampleg\}$

\begin{scaletikzpicturetowidth}{\textwidth}
\begin{tikzpicture}[scale=\tikzscale, font=\tiny]
	\pgfdeclarelayer{background}
	\pgfdeclarelayer{foreground}
	\pgfsetlayers{background,main,foreground}
	
	\begin{pgfonlayer}{main}
		%vertices
		\node (vroot) at (0, 1) [fpi] {\examplefpi{\examplea}{\examplei}{\{\examplea, \examplee, \examplef, \exampleg\}}};

		\node (v00) at (-4, -0) [fpi] {\examplefpi{\examplea}{\examplee}{\{\exampleb, \examplec, \exampled\}}};
		\node (v01) at (4, -0) [fpi] {\examplefpi{\examplee}{\examplei}{\{\examplee, \examplef, \exampleh\}}};

                \node (v10) at (-4, -1) [iis] {\exampleiis{\examplea}{\examplee}{\{\examplea\}}{0}};
                \node (v11) at (2, -1) [fpi] {\examplefpi{\examplee}{\exampleg}{\{\examplee, \examplef\}}};
                \node (v12) at (6, -1) [iis] {\exampleiis{\exampleg}{\examplei}{\{\exampleg\}}{0}};

                \node (v20) at (-4, -2) [iis] {\exampleiis{\examplea}{\examplee}{\{\exampleb, \examplec, \exampled\}}{1}};
                \node (v21) at (6, -2) [iis] {\exampleiis{\exampleg}{\examplei}{\{\exampleh\}}{1}};
		%edges
                \draw (vroot) edge[edge] (v00);
                \draw (vroot) edge[edge] (v01);

		\draw (v00) edge[edge] (v10);
		\draw (v01) edge[edge] (v11);
		\draw (v01) edge[edge] (v12);

		\draw (v10) edge[edge] (v20);
		\draw (v12) edge[edge] (v21);
	\end{pgfonlayer}
	
	\begin{pgfonlayer}{background}
		\draw[-{Triangle[width=30pt,length=17pt,color=gray]}, line width=15pt, color=gray](8, 1) -- (-8, 1);
		\draw[-{Triangle[width=30pt,length=17pt,color=gray]}, line width=15pt, color=gray](-8, -0) -- (8, -0);
		\draw[-{Triangle[width=30pt,length=17pt,color=gray]}, line width=15pt, color=gray](8, -1) -- (-8, -1);
		\draw[-{Triangle[width=30pt,length=17pt,color=gray]}, line width=15pt, color=gray](-8, -2) -- (8, -2);
	\end{pgfonlayer}
\end{tikzpicture}
\end{scaletikzpicturetowidth}

\caption{
An example run of the protocol. $\mathcal{X}_1$ initiates reconciliation for all items between \examplea~ and \examplei (ordered alphabetically) by sending its fingerprint for the whole interval.
Upon receiving this interval fingerprint, $\mathcal{X}_0$ locally computes $\fp{\interval{\examplea}{\examplei}{X_0}}$. Since the result does not match the received interval, $\mathcal{X}_0$ splits $X_0$ into two parts of equal size and transmits interval fingerprints for these subintervals.
In the third round, $\mathcal{X}_1$ locally computes fingerprints for the two received intervals, but neither matches. $\abs{\interval{\examplea}{\examplee}{X_1}} \leq 1$, so $\mathcal{X}_1$ transmits the corresponding interval items set, i.e. $\iis{\examplea}{\examplee}{\examplea}{0}$. $\abs{\interval{\examplee}{\examplei}{X_1}} > 1$, so another recursion step is performed. After splitting the interval, the lower interval is large enough to send its fingerprint, the upper one however only contains one item and thus results in another interval item set.
In the fourth and final communication round, $\mathcal{X}_0$ receives two interval item sets and answers with the items it holds within those intervals. When it receives the interval fingerprint $\ifp{\examplee}{\exampleg}{X_1}$, it computes an equal fingerprint for $\ifp{\examplee}{\exampleg}{X_0}$, so no further action is required for this particular interval. TODO prettify this caption
}

\label{simple-set-reconciliation-example}
\end{figure*}

\subsection{Observations}

Partitioning based on a total order allows the nodes to perform a limited form of queries, i.e. range queries. A node can ask for reconciliation within a certain interval, rather than over the whole universe.

If the universe $U$ is finite, the greatest element of the universe cannot be exchanged, since all ranges have an exclusive upper boundary. We will thus assume that for a universe $U$ of interest, nodes are actually using $\tilde{U} \defeq U \mathbin{\dot{\cup}} \top$ with $u \preceq \top$ for all $u \in U$.

If the universe $U$ is not finite, then there are items that require an arbitrary amount of bytes to encode. Since the protocol needs to transmit items to denote interval boundaries, no reasonably complexity guarantees can be given for infinite universes. We will thus assume $U$ to be finite and small enough that items can be reasonably encoded. This assumption is not very restrictive in practice because nodes can always synchronize hashes of items rather than the items themselves. The protocol can then be either followed by a phase where hashes of interest are transferred and anwered by the actual items, or the protocol can be made aware of the distinction and use hashes as interval boundaries while transmitting actual items for interval item sets.

When reconciling hashes in place of actual items, any semantically interesting order on the items would be replaced by an arbitrary order on the hashes. But rather than using only's the hashes as interval boundaries, one can just as well add additional information. For example if the universe of interest consists of timestamped strings of arbitrary length, the interval boundaries can consist of timestamped hashes, ordered by timestamp first and using the numeric order on the hashes as a tiebreaker. \Cref{set-reconciliation-graphs} gives a more detailed example for utilizing this technique.

\section{Proof of Correctness}
\label{set-reconciliation-simple-correct}

\newcommand{\intcount}[1]{\mathit{count}_{#1}}

We now prove the correctness of the protocol. The protocol is correct if for all $x, y \in U$ both nodes eventually hold $\interval{x}{y}{X_i} \cup \interval{x}{y}{X_j}$ after a node $\mathcal{X}_i$ has received a message pertaining to the interval $[x, y)$.

\begin{caselist}
\case[Interval Item Set] \label{case-iis}  If the message contains the interval item set $\iisnatural{x}{y}{X_j}{0}$, then $\mathcal{X}_i$ adds all items to its set, resulting in $\interval{x}{y}{X_i} \cup \interval{x}{y}{X_j}$ as desired. $\mathcal{X}_j$ then receives $\iis{x}{y}{\interval{x}{y}{X_i} \setminus \interval{x}{y}{X_j}}{1}$, ending up with $\interval{x}{y}{X_j} \cup (\interval{x}{y}{X_i} \setminus \interval{x}{y}{X_j}) = \interval{x}{y}{X_i} \cup \interval{x}{y}{X_j}$ as desired.

\case[Interval Fingerprint] \label{case-ifp} Otherwise, the message contains an interval fingerprint $\ifp{x}{y}{X_j}$.

\begin{caselist}
\case[Equal Fingerprints] If $\fp{\interval{x}{y}{X_j}} = \fp{\interval{x}{y}{X_i}}$, the protocol terminates immediately and no changes are performed by any node. Assuming no fingerprint collision occurred, $\interval{x}{y}{X_i} = \interval{x}{y}{X_j} = \interval{x}{y}{X_i} \cup \interval{x}{y}{X_j}$ as desired.

\case[Recursion Anchor] \label{case-ifp-anchor} If $\mathcal{X}_i$ adds the interval item set $\iisnatural{x}{y}{X_i}{0}$, then \cref{case-iis} applies when the other node receives the response, with the roles reversed.

\case[Recurse] Let $\intcount{i} \defeq \abs{\interval{x}{y}{X_i}}$ and $\intcount{j} \defeq \abs{\interval{x}{y}{X_j}}$. $\intcount{j} \geq 2$, since otherwise $\mathcal{X}_j$ would have sent an item set for the interval. Similarly, $\intcount{i} \geq 2$, since we are not in \cref{case-ifp-anchor}. Thus, $\intcount{i} + \intcount{j} \geq 4$, and the protocol has already been proven correct for all cases where $\intcount{i} + \intcount{j} < 4$. 

We can thus finish the proof by induction on $\intcount{i} + \intcount{j}$, using the induction hypothesis that for all $x', y' \in U$ such that $\abs{\interval{x'}{y'}{X_i}} + \abs{\interval{x'}{y'}{X_j}} < \intcount{i} + \intcount{j}$ the protocol correctly reconciles $\interval{x'}{y'}{X_i}$ and $\interval{x'}{y'}{X_j}$.

$\mathcal{X}_i$ partitions the interval into $k \geq 2$ subintervals, of which at least two must be nonempty.
Thus $\abs{\interval{m_l}{m_{l + 1}}{X_i}} < \intcount{i}$ for all $0 \leq l < k$.
Furthermore, $\interval{m_l}{m_{l + 1}}{X_j} \subseteq \interval{x}{y}{X_j}$ and thus $\abs{\interval{m_l}{m_{l + 1}}{X_j}} \leq \abs{\interval{x}{y}{X_j}}$, so overall we have $\abs{\interval{m_l}{m_{l + 1}}{X_i}} + \abs{\interval{m_l}{m_{l + 1}}{X_j}} < \intcount{i} + \intcount{j}$ and can apply the induction hypothesis to conclude that every subinterval is correctly reconciled. Since the subintervals partition the original interval, the original interval is then correctly reconciled as well.
\end{caselist}
\end{caselist}

\section{Complexity Analysis}
\label{set-reconciliation-complexity}

The protocol gives nodes the freedom to respond to an interval fingerprint with an interval item set even if the interval fingerprint is arbitrarily large. For a meaningful complexity analysis we need to restrict the behavior of the node, a realistic modus operandi is for a node to send an interval item set whenever it holds a number of items less than or equal to some threshold $t \in \mathbb{N}, t \geq 1$ within the interval. Higher choices for $t$ reduce the number of roundtrips, but increase the probability that a items is being sent even though the other node already holds it.

A node is similarly given freedom over the number of subintervals into which to split an interval when recursing. We will assume a node always splits into at most $b \in \mathbb{N}, b \geq 2$ subintervals. As with $t$, higher numbers reduce the number of roundtrips at the cost of potentially sending items or fingerprints that did not need sending.

Because we want to analyze not only the worst-case complexity but also the complexity depending on the similarity between the two sets held by the participating nodes, we define some rather fine-grained instance size parameters: $n_0$ and $n_1$ denote the number of items held by $\mathcal{X}_0$ and $\mathcal{X}_1$ respectively. We let $n \defeq n_0 + n_1$, $n_{min} \defeq \mathit{min}(n_0, n_1)$, $n_{max} \defeq \mathit{max}(n_0, n_1)$, $n_{\cap} \def \abs{\interval{x}{y}{X_0} \cap \interval{x}{y}{X_1}}$, $n_{\cup} \def \abs{\interval{x}{y}{X_0} \cup \interval{x}{y}{X_1}}$ and $n_{\triangle} \defeq \abs{(\interval{x}{y}{X_0} \cup \interval{x}{y}{X_1}) \setminus (\interval{x}{y}{X_0} \cap \interval{x}{y}{X_1})}$. TODO remove those that are not needed

\subsection{Preliminary Observations}

A helpful observation for the following analysis is that the interval fingerprints that are being exchanged during a protocol run form a rooted tree where every vertex has at most $b$ children. When a leaf of the tree is reached, an exchange of interval item sets follows. Equal fingerprints can also cut the tree short, but for the following worst-case analyses we will assume this does not occur.

Node $\mathcal{X}_i$ can branch at most $\ceil{\mathit{log}_{b}(n_i)}$ times, so the overall height of the tree is bounded by $2 \cdot\ceil{\mathit{log}_{b}(n_{min})}$. The number of vertices of such a complete tree of height $h$ is at most $\sum_{i=0}^{h} b^{i} = \frac{b^{h} - 1}{b - 1}$. For $h \leq 2 \cdot\ceil{\mathit{log}_{b}(n_{min})}$, $\frac{b^{h} - 1}{b - 1} \leq 2 \cdot 2 \cdot n_{min} \leq 2n \in \complexity{n}$.

The parameter $t$ determines when recursion is cut off, and thus influences the height of the tree. For $t = 1$, the protocol recurses as far as possible. For $t = b$, the last level of recursion is cut off, for $t = b^2$ the last two levels, and so on. Overall, the height of the tree is reduced by $\floor{\mathit{log}_{b}(t)}$.

\subsection{Communication Rounds}

The number of communication rounds clearly corresponds to the height of the tree, plus $2$ to account for the exchange of interval item sets, so the worst-case is $2 + 2 \cdot\ceil{\mathit{log}_{b}(n_{min})} - \floor{\mathit{log}_{b}(t)} \in \complexity{\mathit{log}_{b}(n)}$. This number cannot be bounded by $n_{\triangle}$, as witnessed by problem instances where one node is missing exactly one item compared to the other node. In such an instance, $b - 1$ branches in each recursion step result in equal fingerprints, but the one branch that does continue reaches the recursion anchor only after the full number of rounds. See \cref{fig:worst-rounds} for an example.

\begin{figure*}
$X_0 \defeq \{\examplea, \exampleb, \examplec, \exampled, \examplee, \exampleg, \exampleh \}$
\hfill
$X_1 \defeq \{\examplea, \exampleb, \examplec, \exampled, \examplee, \examplef, \exampleg, \exampleh \}$

\begin{scaletikzpicturetowidth}{\textwidth}
\begin{tikzpicture}[scale=\tikzscale, font=\tiny]
	\pgfdeclarelayer{background}
	\pgfdeclarelayer{foreground}
	\pgfsetlayers{background,main,foreground}
	
	\begin{pgfonlayer}{main}
		%vertices
		\node (vroot) at (0, 1) [fpi] {\examplefpi{\examplea}{\examplei}{\{\examplea, \exampleb, \examplec, \exampled, \examplee, \examplef, \exampleg, \exampleh\}}};

		\node (v00) at (-4, -0) [fpi] {\examplefpi{\examplea}{\examplee}{\{\examplea, \exampleb, \examplec, \exampled\}}};
		\node (v01) at (4, -0) [fpi] {\examplefpi{\examplee}{\examplei}{\{\examplee, \exampleg, \exampleh\}}};

                \node (v10) at (2, -1) [fpi] {\examplefpi{\examplee}{\exampleg}{\{\examplee, \examplef\}}};
                \node (v11) at (6, -1) [fpi] {\examplefpi{\exampleg}{\examplei}{\{\exampleg, \exampleh\}}};

                \node (v20) at (2, -2) [iis] {\exampleiis{\examplee}{\exampleg}{\{\examplee\}}{0}};

                \node (v30) at (2, -3) [iis] {\exampleiis{\examplee}{\exampleg}{\{\examplef\}}{1}};

		%edges
                \draw (vroot) edge[edge] (v00);
                \draw (vroot) edge[edge] (v01);

		\draw (v01) edge[edge] (v10);
		\draw (v01) edge[edge] (v11);

		\draw (v10) edge[edge] (v20);
		
		\draw (v20) edge[edge] (v30);
	\end{pgfonlayer}
	
	\begin{pgfonlayer}{background}
		\draw[-{Triangle[width=30pt,length=17pt,color=gray]}, line width=15pt, color=gray](8, 1) -- (-8, 1);
		\draw[-{Triangle[width=30pt,length=17pt,color=gray]}, line width=15pt, color=gray](-8, -0) -- (8, -0);
		\draw[-{Triangle[width=30pt,length=17pt,color=gray]}, line width=15pt, color=gray](8, -1) -- (-8, -1);
		\draw[-{Triangle[width=30pt,length=17pt,color=gray]}, line width=15pt, color=gray](-8, -2) -- (8, -2);
		\draw[-{Triangle[width=30pt,length=17pt,color=gray]}, line width=15pt, color=gray](8, -3) -- (-8, -3);
	\end{pgfonlayer}
\end{tikzpicture}
\end{scaletikzpicturetowidth}

\caption{An example run of the protocol that takes the greatest possible number of rounds even though $n_{\triangle} = 1$. $b \defeq 2, t \defeq 1$.}

\label{fig:worst-rounds}

\end{figure*}

\subsection{Communication Complexity}

The total number of bits that needs to be transmitted during a protocol run is proportional to the number of vertices in the tree. Every interval fingerprint consists of two items and one fingerprint, so assuming $U$ is finite this lies in $\complexity{1}$. Since there are at most $2n$ vertices in the tree, the interval fingerprints require at most $\complexity{n}$ bits to be communicated.

The exchange of interval item sets consists in the worst case of exchanging every item using $\ceil{\frac{n}{t}}$ interval item sets. An interval item set needs to transmit two items to encode the boundaries, as well as the items themselves, which lies in $\complexity{1}$ per interval item set. All interval item sets together thus amount to another $\complexity{n}$, leading to a total of $\complexity{n}$ bits being transmitted in the worst case.

\Cref{fig:worst-bytes} shows a worst-case example in which the tree of height $h \defeq \mathit{log}_{b}(2 \cdot n_{min})$ has all $\frac{b^{h} - 1}{b - 1}$ vertices. 

TODO bound complexity by difference, also average case

\begin{figure*}
$X_0 \defeq \{\examplea, \examplec, \examplee, \exampleg \}$
\hfill
$X_1 \defeq \{\examplea, \exampleb, \examplec, \exampled, \examplee, \examplef, \exampleg, \exampleh\}$

\begin{scaletikzpicturetowidth}{\textwidth}
\begin{tikzpicture}[scale=\tikzscale, font=\tiny]
	\pgfdeclarelayer{background}
	\pgfdeclarelayer{foreground}
	\pgfsetlayers{background,main,foreground}
	
	\begin{pgfonlayer}{main}
		%vertices
		\node (vroot) at (0, 1) [fpi] {\examplefpi{\examplea}{\examplei}{\{\examplea, \exampleb, \examplec, \exampled, \examplee, \examplef, \exampleg, \exampleh\}}};

		\node (v00) at (-4, -0) [fpi] {\examplefpi{\examplea}{\examplee}{\{\examplea, \examplec\}}};
		\node (v01) at (4, -0) [fpi] {\examplefpi{\examplee}{\examplei}{\{\examplee, \exampleg\}}};

                \node (v10) at (-6, -1) [fpi] {\examplefpi{\examplea}{\examplec}{\{\examplea, \exampleb\}}};
                \node (v11) at (-2, -1) [fpi] {\examplefpi{\examplec}{\examplee}{\{\examplec, \exampled\}}};
                \node (v12) at (2, -1) [fpi] {\examplefpi{\examplee}{\exampleg}{\{\examplee, \examplef\}}};
                \node (v13) at (6, -1) [fpi] {\examplefpi{\exampleg}{\examplei}{\{\exampleg, \exampleh\}}};

                \node (v20) at (-6, -2) [iis] {\exampleiis{\examplea}{\examplec}{\{\examplea\}}{0}};
                \node (v21) at (-2, -2) [iis] {\exampleiis{\examplec}{\examplee}{\{\examplec\}}{0}};
                \node (v22) at (2, -2) [iis] {\exampleiis{\examplee}{\exampleg}{\{\examplee\}}{0}};
                \node (v23) at (6, -2) [iis] {\exampleiis{\exampleg}{\examplei}{\{\exampleg\}}{0}};

                \node (v30) at (-6, -3) [iis] {\exampleiis{\examplea}{\examplec}{\{\exampleb\}}{1}};
                \node (v31) at (-2, -3) [iis] {\exampleiis{\examplec}{\examplee}{\{\exampled\}}{1}};
                \node (v32) at (2, -3) [iis] {\exampleiis{\examplee}{\exampleg}{\{\examplef\}}{1}};
                \node (v33) at (6, -3) [iis] {\exampleiis{\exampleg}{\examplei}{\{\exampleh\}}{1}};
		%edges
                \draw (vroot) edge[edge] (v00);
                \draw (vroot) edge[edge] (v01);

		\draw (v00) edge[edge] (v10);
		\draw (v00) edge[edge] (v11);
		\draw (v01) edge[edge] (v12);
		\draw (v01) edge[edge] (v13);

		\draw (v10) edge[edge] (v20);
		\draw (v11) edge[edge] (v21);
		\draw (v12) edge[edge] (v22);
		\draw (v13) edge[edge] (v23);

		\draw (v20) edge[edge] (v30);
		\draw (v21) edge[edge] (v31);
		\draw (v22) edge[edge] (v32);
		\draw (v23) edge[edge] (v33);
	\end{pgfonlayer}
	
	\begin{pgfonlayer}{background}
		\draw[-{Triangle[width=30pt,length=17pt,color=gray]}, line width=15pt, color=gray](8, 1) -- (-8, 1);
		\draw[-{Triangle[width=30pt,length=17pt,color=gray]}, line width=15pt, color=gray](-8, -0) -- (8, -0);
		\draw[-{Triangle[width=30pt,length=17pt,color=gray]}, line width=15pt, color=gray](8, -1) -- (-8, -1);
		\draw[-{Triangle[width=30pt,length=17pt,color=gray]}, line width=15pt, color=gray](-8, -2) -- (8, -2);
		\draw[-{Triangle[width=30pt,length=17pt,color=gray]}, line width=15pt, color=gray](8, -3) -- (-8, -3);
	\end{pgfonlayer}
\end{tikzpicture}
\end{scaletikzpicturetowidth}

\caption{An example run of the protocol that requires transmitting the maximum amount of bytes. $b \defeq 2, t \defeq 1$.}

\label{fig:worst-bytes}
\end{figure*}

\subsection{Computational Complexity}

We now analyze the computational cost incurred by a single communication round, i.e. computing the response to a message. This includes both fingerprint comparisons as well as locating the items to transmit. We do however assume that an auxiliary data structure is available to help with this computation, e.g. the a fingerprint tree structure presented in \cref{TODO}. We exclude both space usage and maintenance cost for this data structure from the per-round complexity.

We will assume that transferring an item as part of an interval item set requires $\complexity{1}$ time and space. The relevant computational overhead per communication round thus consists of computing $\fp{\interval{x}{y}{X_i}}$ for every received interval fingerprint $\ifp{x}{y}{X_j}$, as well as partitioning $\interval{x}{y}{X_i}$ in case of a mismatch and computing the fingerprints over all subintervals. These computations can be performed independently for all received interval fingerprints, so in particular they can be performed sequentially, reusing space. The overall space complexity of the per-round computations is equal to that of the computations for a single interval fingerprint.

A naive approach is to query the auxiliary data structure for each received interval fingerprint individually. The maximum amount of queries is upper-bounded by $n$, it is certainly impossible to receive more than $n$ interval fingerprints within a single communication round. Unfortunately, the greatest possible number of interval fingerprints within a single round corresponds to the number of leaves in the recursion tree, which is in $\complexity{n}$. The auxiliary data structure requires $\complexity{\mathit{log}(n)}$ time, leading to an overall $\complexity{n \cdot \mathit{log}(n)}$.

We can bring this down to $\complexity{\mathit{log}(n) + n_{\triangle}}$ by augmenting the auxiliary data structure with parent-pointers to allow efficient in-order traversal, and by memoizing some intermediate fingerprint computation results. TODO (write fingerprint chapter first)

\section{Smaller Optimizations}
\label{set-reconciliation-simple-optimizations}

We now give a list of optimizations which do not impact the overall complexity analysis, but which do improve on some constant factors.

\subsection{Non-Uniform Partitions}

When partitioning an interval into subintervals, the protocol does not specify where exactly to place the boundaries. Splitting into partitions of roughly equal sizes makes a lot of sense if new data could arise anywhere within the linear order with equal probability. Is however the items are likely to fall within certain ranges of the order, it can be more efficient to use more fine-grained partitions within those regions. If for example items are sorted by timestamp, and new items are expected to be propagated to every node in the system within a couple of seconds, then all items older than ten seconds can be comfortably lumped together in a large interval.

\subsection{Subset Checks}

When a node $\mathcal{X}_i$ receives an interval fingerprint $\ifp{x}{y}{X_j}$, it might have a different fingerprint for that interval, but one of the resulting subintervals could match the received fingerprint. The node can then ignore that interval, transmitting only the remaining ones.

This scenario is a special case of receiving the fingerprint of a subset of the item one holds within an interval. In principle, a node can compute the fingerprints of arbitrary subsets of its interval, trying to find a match. If a match is found, the node knows that it holds a superset of the items the other node holds within the interval. The protocol could be extended with a message part that transmits items and does not warrant a response, this would be used to transmit $\interval{x}{y}{X_i} \setminus \interval{x}{y}{X_j}$.

Computing the fingerprint of all subsets of interval is infeasible since they are $2^n$ many subsets. As discussed in \cref{TODO} however, if the group operation for fingerprint computation is xor, a node can check whether the fingerprint of a subset of items matches a given fingerprint in $\complexity{n^3}$.
Assuming no fingerprint collisions occurred, checking whether a subset of $\interval{x}{y}{X_i}$ matches $\fp{\interval{x}{y}{X_j}}$ is equivalent to checking whether $\interval{x}{y}{X_i}$ is a superset of $\interval{x}{y}{X_j}$.

This leads to a protocol with a per-round computational complexity of $\complexity{n^3}$ and the same worst-case guarantees, but which can skip recursive steps more often than the basic protocol. In particular, whenever two nodes reconcile where one of them holds a subset of the items of the other node, reconciliation terminates after at most three communication rounds.

TODO precomputation, see comment %https://piazza.com/class_profile/get_resource/jl30kcwntnn7f1/jlh5nh8gwuh11s
TODO: this is O(1), not polynomial?

\subsection{Efficiently Encoding Intervals}

A naive encoding of interval fingerprints and interval item sets would transmit both boundaries of every interval, transmitting $2n$ items if there are $n$ intervals in a message. This can be brought down to $1 + n$ by utilizing that the lower boundary of all but the first interval it is also the upper boundary of the preceding one. An efficiently encoded message consists of the number of intervals it contains, followed by the lower boundary of the first interval, followed by pairs of interval information and the upper boundary of the interval that the information pertains to, where each \defined{interval information} is either a fingerprint, a set of items, or a dummy value signaling that this part of the overall interval is already fully synchronized.

Note additionally that two adjacent interval item sets can be merged into a single one, saving on the transmission of the boundary between them.

\subsection{Utilizing Interval Boundaries}

Whenever the lower boundary of an interval is transmitted, an item is transmitted. If the receiving node knew whether the other node held this item, it would automatically be reconciled and could be excluded from further recursion steps. One way for achieving this is to tag each interval with a bit to indicate whether the lower boundary is being held by the sender. A different approach is to require that intervals are only split at items which the splitting node holds, this way no explicit bits needs to be transmitted, yet many boundaries can be identified as being held by the other node.

Recall that we assume $U$ to be finite and with the greatest element $\top$, which never occurs as the lower boundary of an interval. For $x \in U \setminus \{\top\}$ we can thus denote by $\successor{x}$ the unique $z \in U$ such that $x \prec z$ and there exists no $y \in U$ such that $x \prec y \prec z$. When sending or receiving an interval from $x$ to $y$ for which both nodes know that the sending node holds $x$, both nodes act as if the interval went from $\successor{x}$ to $y$, except that the receiving node adds $x$ to its local set of items.

\subsection{Multi-Level Fingerprints}

Since the protocol interprets equal interval fingerprints as both nodes holding the same set of items within the interval but never verifies this merely probabilistic assumption, fingerprints need to be long enough to guarantee low collision probability. The longer the fingerprints, the more bits need to be transmitted however.

One can use smaller fingerprints if in case of equal fingerprints the nodes then exchange an additional fingerprint rather than immediately stopping the recursion for this interval. Whenever a node receives a first-level interval fingerprint that is equal to its own, it answers with the second-level fingerprint for the same interval. When a node receives a second-level interval fingerprint that is equal to its own, it terminates the recursion. If it is not equal, it recurses as usual, in particular it sends first-level fingerprints for any large subintervals.

This scheme can of course be extended to an arbitrary number of fingerprint levels. Every level increases the worst-case number of roundtrips by one, but decreases the average message size. If cryptographically secure fingerprints are desired, low levels of the fingerprint hierarchy do not have to be secure, only the top level fingerprint needs to be so. Alternatively, each fingerprint level can consist of some substring of a long, cryptographically secure fingerprint, such that the concatenation of these substrings yields the original fingerprint.

% TODO compression, utilizing that boundaries are ordered. Simple approach: difference to next boundary
% probably need to understand section 6.2 of https://research.cyber.ee/~peeter/research/JCS161.pdf

\section{Reconciling Hash Graphs}
\label{set-reconciliation-graphs}

In this section, we demonstrate how to apply the set reconciliation protocol the problem of reconciling hash graphs. Hash graphs arise in contexts where pieces of data refer to other pieces of data by a secure hash computed over the data to be referred to. Distributed version control systems such as git~\cite{chacon2014pro} for example represent the evolving contents of a directory as a set of deltas which describe how the contents changed from an earlier version, this earlier version is referenced as the hash of another such object. Some objects describe how to merge conflicting concurrent changes, these objects can reference multiple other objects. Since addressing uses a secure hash function, an object can only reference objects which existed prior to it. All in all, objects can thus form arbitrary directed, acyclic graphs (dags).

When two nodes wish to synchronize, they update each other's histories to the union of all objects known to both of them. This is a natural setting for the set reconciliation protocol. Since objects can become arbitrarily large, one would reconcile sets of hashes of these objects, and then a second stage request the actual objects for all newly obtained hashes.

This approach completely ignores the edges of the dag. While at first glance it might seem inefficient to ignore available structural information, the fact that the dags can take arbitrary shapes makes it hard to utilize the edge structure. The graphs might be dense or sparse, could contain arbitrarily large independent sets, paths, tournaments etc.

Even though arbitrary dags are possible and thus the logarithmically bounded worst-case complexity is important, one can in practice make some reasonable assumptions about the hash graphs arising from an average, version-controlled repository. It seems likely for example that most concurrent new work is performed relative to a rather recent state of the repository, whereas work based off a very old state is rather unlikely. We could hope for better average reconciliation times if the reconciliation algorithm leveraged this expectation.

To that end, rather than reconciling merely hashes of objects, we can reconcile pairs $(\mathit{depth}, \mathit{hash})$, sorting by depth first and hash second. The \defined{depth} of an object is the length of the longest path from that object to a root object, i.e. an object without predecessors. That is, the depth of an object is $1$ greater than the greatest depth of any predecessor object. If most concurrent work is based off similarly new state, then it also falls into a similar interval of the linear order. A protocol run does not need to recurse into intervals of low depth then, since no concurrently working peer produces new objects of low depth. If this assumption turns out to be wrong, the protocol nevertheless upholds its worst-case guarantee of a logarithmic number of communication rounds.

Another common scenario is that a node has not produced any new local objects and merely wants to catch up with the current state. Let $d$ be the maximal depth among all objects the node holds. Then rather than reconciling the interval from $(0, 0)$ to $\top$, the node can reconcile the interval from $(0, 0)$ to $(d + 1, 0)$ as well as sending an empty interval item set from $(d + 1, 0)$ to $\top$. All new changes of depth $d + 1$ or greater are then fetched in only two communication rounds. Any concurrent work (from a depth perspective) is reconciled as usual, and if all new changes are based off a recent version, the whole reconciliation process only takes a single round trip.

\chapter{Protocol Variants}
\label{other-data-structures}
% !TEX root = ../main.tex

In this chapter, we sketch some adaptations of the set reconciliation algorithm to related problems. \Cref{general-partitions} considers more general partitioning strategies than one-dimensional ranges. \Cref{maps} presents how to reconcile key-value mappings. \Cref{set-mirror} shows an asymmetric variant of the protocol, mirroring the contents of a set or map held at a primary node to a replica node. \Cref{authenticated} demonstrates how our data structures for fingerprint computations can also serve as simple authenticated data structures.

\section{Higher-Dimensional Ranges}
\label{general-partitions}

The set reconciliation algorithm recursively partitions the item sets of both nodes by splitting the sets into successive ranges. But for the correctness of the algorithm, the exact mechanism by which partitions are chosen is irrelevant. All that is necessary is that in each communication round, when receiving a mismatching fingerprint for some non-empty $S \subseteq U$, the node chooses a partition $\disjointunion{S_0}{\disjointunion{S_1}{\disjointunion{\ldots}{S_k}}} = S$ of $S$, and then communicates to the other node the choice of the partition and how its items are distributed among the individual sets.

Partitioning into successive ranges over a linearly ordered set has some nice properties: It can be applied to virtually any set since one can always arbitrarily define a linear order. The choice of the partition and the way the items are distributed among the sets can be encoded and transmitted efficiently. And finally, when receiving an arbitrarily chosen partition into successive ranges, a node can always efficiently compute the items it holds within that partition as well as the fingerprint over these items.

As long as nodes always wants to synchronize the complete sets they hold, this is completely sufficient. But more elaborate partitioning strategies can become attractive if nodes might wish to only synchronize certain subsets. The partitioning mechanism effectively dictates which such subsets can be described, taking on the role of a query language. A natural next step is to look for more powerful partitioning mechanisms that still admit efficient fingerprint computation.

We now give one such partitioning scheme by generalizing the one-dimensional ranges of the original presentation of the algorithm to $k$-dimensional ranges. Let $U$ be a finite set of items, and let $\preceq_0, \preceq_1, \ldots, \preceq_{k-1}$ be linear orders over $U$. A \defined{$k$-dimensional range} is a $k$-tuple $((x_0, y_0), (x_1, y_1), \ldots, (x_{k-1}, y_{k-1}))$ with $x_i \preceq_i y_i$. Given some $S \subseteq U$, the items from $S$ in this range are all items $v \in S$ such that $x_i \preceq_i v \preceq_i y_i$ for all $0 \leq i < k$. Note that for $k = 1$ this is equivalent to the ranges used in previous chapters. The reconciliation protocol needs to be adapted so that messages include the $k$-dimensional range boundaries, but otherwise no changes are required.

Fingerprints can be computed efficiently by storing the set as a (balanced) $k$-d tree~\cite{bentley1975multidimensional}. As $k$-d tree are binary trees, the labels from \cref{fingerprints} can be used without any modification. Fingerprint computation still works by traversing the tree, alternating between the dimensions just as in e.g. a $k$-d tree item lookup. For $k = 1$, this yields exactly the algorithms of \cref{fingerprints}.

Generalized partitioning schemes, including $k$-dimensional ranges, can not only be applied to the regular set reconciliation protocol, but also to all modifications discussed in the further sections.

\section{Map Reconciliation}
\label{maps}

A key-value mapping, i.e. a partial function $\partialfun{\m}{K}{V}$ with a finite domain from some set of keys $K$ to a set of values $V$, can be reconciled by reconciling the set $\set{(k, v) \mid \text{$k \in \domain(\m)$ and $v = \m(k)$}}$. After this reconciliation, a node may have obtained two pairs $(k, v), (k, v')$ if the two nodes mapped the same key $k$ to distinct values $v, v'$, so the resulting set would not correspond to an updated map. In those cases, both nodes compute the single new image of $k$ as $\f(v, v')$, where $\fun{\f}{V \times V}{V}$ is some function known to all participating nodes.

Particularly interesting are cases where $\f$ can be computed via another interactive protocol. If for example $V$ consists of the fingerprints of finite subsets of some universe $U$, and $\f(v, v')$ is defined to be the fingerprint of the union of the two sets whose fingerprints are $v$ and $v'$, then the two nodes can run a set reconciliation session to efficiently obtain the union. Viewn more abstractly, when reconciling a map, the values can be reconciled via arbitrary nested protocol invocations.

\section{Set and Map Mirroring}
\label{set-mirror}

We based the presentation of our synchronization approach on set reconciliation because it is a symmetric problem where both node use identical algorithms to compute identical types of messages. A related, asymmetric problem is that of a \defined{replica} node setting its locally stored set to that of a \defined{primary} node, utilizing similarity between the two initial sets to minimize the communication complexity. The approach of recursively exchanging fingerprints for subsets of decreasing size can be modified to solve this \defined{set mirroring} problem.

The primary node can run exactly the same protocol as that for such reconciliation. The replica node uses a slightly modified version. Whenever it receives a range item set, it adds the received items to its local set as usual, but then it deletes all item it holds within that range which were not part of the received range item set. Whenever the replica node sends a range item set, it sends an empty one.

The complexity analysis is identical to that of the set reconciliation protocol, the correctness argument is analogous: ranges with equal fingerprints are already correctly mirrored if no fingerprint collisions occurred, exchanging range item sets correctly mirrors all items within that range, and large ranges with non-equal fingerprints can be handled recursively because mirroring the partitions of a set results in mirroring the whole set.

Just as for reconciliation, maps can be mirrored by interpreting them as sets of key-value pairs. Mirroring maps has some interesting use cases, for example filesystems can be regarded as maps from paths to strings. The efficient creation of backups then becomes the problem of mirroring such a map onto a backup server that may already hold a similar, older backup. An equivalent problem is that of efficiently distributing source code updates from a server to clients which may hold old versions of the source code.

\section{Authenticated Data Structures}
\label{authenticated}

Authenticated data structures solve the problem of outsourcing data structure membership queries processing to untrusted replicas rather than processing all queries at the original trusted data source. The trusted source publishes a short digest to a client. The client can then send a query to a replica, which answers with the result and a small certificate which together with the digest proves that the answer is indeed correct. For more details on this three-party model and pointers to the rich literature on the topic, we refer to~\cite{martel2004general}.

Merkle trees form a simple authenticated set. The root label is the digest, and the certificate for an affirmative membership query consists of the labels of the children of the vertices on the path from the root to the item in question. These labels can be used to recompute the root label, proving that the item is part of the original tree. Fabricating a sequence of labels to fake inclusion of an item amounts to breaking the hash function. Non-membership queries can be authenticated by providing certificates for tree membership of items stored in adjacent leaves such that one is less than and one is greater than the item in question.

Our examination of randomized data structures was driven by the need for Merkle-like properties, so it is not very surprising that they can also be used as authenticated data structures. Indeed~\cite{naor2000certificate} mentions treaps as an alternative to regular Merkle trees. Monoidal labels based on Cayley hash functions remove the need for a specific tree shape to be maintained. Skip lists as authenticated data structures have been studied in~\cite{goodrich2000efficient}, but they require the hash function to be commutative. Our construction removes this assumption.


\chapter{Related Work}
\label{related-work}
% !TEX root = ../main.tex

\begin{itemize}
  \item set reconciliation literature
  \item hash graph synchronization
  \item filesystem synchronization
  \item history-based synchronization
\end{itemize}

%The literature for set reconciliation is fixated on minimizing roundtrips, at the cost of high computation times. \cite{minsky2003set} gives theoretical limits and a very clever protocol approaching them, but which requires $O(n^3)$ computation time per round-trip. The authors acknowledge the practical infeasibility and offer \cite{minsky2002practical}, which is also the only paper that cares whether auxiliary data structures can be efficiently synchronized with the set to reconcile as it changes. They use a simpler approach highly related to error-correcting codes: Both peers send a digest, if the digests are similar enough, the union can be computed from holding both digests and one of the sets. If they are not similar enough, the set has been too large, so they recursively reconcile partitions of the set. Unfortunately, their auxiliary data structure is not self-balancing, so their complexity guarantees degrade as the set to reconcile is being modified.
%
%More recent work such as \cite{eppstein2011s} or \cite{ozisik2019graphene} focuses on invertible bloom filters, and fully embraces taking $O(n)$ computation time per synchronization session. The probabilistic guarantees also involve enough math with actual numbers to require some healthy suspicion.
%
%All of the previous work assumes that the items to be synchronized all have equal and relatively small size, which our approach does not require. None of the literature approaches utilize any structure of the data, whereas we can easily reconcile certain subsets (e.g. all data from within a timeframe if the total order being used sorts by timestamp).
%
%I am not aware of any literature at all that acknowledges the fact that the participating peers only have finite memory available, particularly a server which synchronizes with many peers concurrently has to enforce low memory consumption per session. Any implementation of a protocol not acknowledging this will simply crash at some point.
%
%Filesystem synchronization literature usually focuses on variants of rsync to optimize the single-file case, efficiently determining which files need updating is rarely discussed. Practical implementations usually sort all filenames, concatenate them, and then run their particular rsync the variant on that string. Using map synchronization should be more efficient.
%
%The basic idea of adding up hashes in a tree is not particularly original, e.g. the CCNx  0.8 Sync protocol~\cite{shang2017survey} does the same to solve a specific goal in a specific context.  This thesis highlights the concept as a standalone algorithmic solution to a general problem, examines the concept more closely (discussing choice of the group operation, security issues, etc.), adapts it to deal with bounded memory, and applies it to more data structures than just sets.

\chapter{Conclusion}
\label{conclusion}
% !TEX root = ../main.tex

We have presented range-based set synchronization, an approach that can reconcile two sets in $\complexity{\log(n)}$ communication rounds, transmitting $\complexity{\min(n_{\triangle} \cdot \log_b(n), n)}$ bits, and performing computations in each communication round requiring $\complexity{\min(k \cdot \log(\abs{S}), \abs{S})}$ time and $\complexity{1}$ space. Other approaches in the literature improve on the communication complexity by removing the logarithmic factor, but at the cost of at least $\complexity{n}$ time and $\complexity{n_{\triangle}}$ space for the per-round computations. The range-based approach is easily adapted to mirror a set, none of the other set reconciliation literature can be modified in this way.

Low computational complexity and bounded space usage are required for robust and scalable implementations, transmitting logarithmically more bits is still more effective in practice than running out of memory and crashing. We thus believe that range based synchronization should be part of the toolkit for building distributed systems just like the more complex reconciliation protocols. As an extreme example, a microcontroller connected to network-attached storage could perform reconciliation of large sets even though it has restrictive computational capabilities.

For maximized efficiency in the face of constrained computational resources, hybrid approaches can be considered. The range-based approach transmits range item sets if the number of items inside a range falls below a threshold $b$. The simple exchange of these items can be considered as a minor optimization on the trivial reconciliation protocol, but any other set reconciliation protocol could also be used. Range-based fingerprint comparisons can thus be used to find ranges containing less than $b$ items for both nodes, these ranges can then be reconciled using e.g. characteristic polynomial interpolation or invertible bloom lookup tables. Note however that if the space usage of the computations is to remain in $\complexity{1}$, these ``leaf'' reconciliations have to be performed sequentially, not in parallel.

\section{Future Work}

Beyond the scope of this thesis, there are some natural further questions to study. \Cref{crypto-general} lists some qualitative arguments for the validity of the approach even in the presence of a small number of adversarially created hash collisions. To give full confidence, a quantitative cryptographic analysis needs to answer questions such as what the number of pairwise distinct collisions that can be adequately protected against through range boundary randomization is, how well range boundary randomization can protect against $k$ different subsets all hashing to the same digest, and whether known attacks against the collision resistance of the relevant monoids can be extended to produce such $k$-wise collisions.

We have considered the range-based approach for a two-party setting only, a natural generalization is that of multi-party set synchronization. This can of course be reduced to pairwise synchronization, but an interesting question is whether more efficient range-based protocols can be developed, similar to how~\cite{boral2014multi} adapts characteristic polynomial interpolation and~\cite{mitzenmacher2018simple} adapts invertible bloom lookup tables.

While we have studied the protocol from a theoretical perspective, there are also some software design issues that arise in the context of actual implementations. Consider for example one of the original motivating examples, an unordered peer-to-peer pubsub mechanism. A node in such a network would ideally perform set reconciliation as a first step in a session, and then eagerly forward new items as they became available. Thus new items might arrive in the middle of another reconciliation step, the same problem occurs when doing multiple reconciliation steps in parallel. The new items need to be buffered until reconciliation has finished, and then they can be transmitted to the peer. One solution is to implement the sets as persistent data structures, but are there more efficient mutable solutions based on locking? Or can the buffering of concurrently arriving items possibly avoid it altogether? We have provided the theoretical basis for robust, scalable systems based on set replication, but a practical implementation raises interesting questions in its own right.


\chapter*{Work Plan}

\begin{itemize}
\item by 05.05: basic set reconciliation chapter
\item by 26.05: fingerprint chapter
\item by 16.06: bounded-memory set reconciliation chapter
\item by 07.07: other data structures chapter
\item by 28.07: conclusion, coherence, polishing
\item 15.08: self-inflicted soft deadline, unless adding more content
\end{itemize}

Possibly a chapter discussing more specifics that would occur when using set reconciliation as the core of an unordered p2p pubsub mechanism.

\bibliographystyle{alphaurl}
\bibliography{main}

\end{document}